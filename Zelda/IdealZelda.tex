\documentclass[11pt, oneside]{article}   	% use "amsart" instead of "article" for AMSLaTeX format
\usepackage{geometry}                		% See geometry.pdf to learn the layout options. There are lots.
\geometry{letterpaper}                   		% ... or a4paper or a5paper or ... 
%\geometry{landscape}                		% Activate for for rotated page geometry
%\usepackage[parfill]{parskip}    		% Activate to begin paragraphs with an empty line rather than an indent
\usepackage{graphicx}				% Use pdf, png, jpg, or eps� with pdflatex; use eps in DVI mode
								% TeX will automatically convert eps --> pdf in pdflatex		
\usepackage{amssymb}
\usepackage{fullpage}
\title{The Ideal Legend of Zelda}
\author{Jason Bell: A Zelda Fan}
%\date{}							% Activate to display a given date or no date

\begin{document}
\maketitle
\section{Context}
%\subsection{}

I love Zelda tremendously.  I do, however, have a few ideas for improvements.    

My ideal Zelda game would have:
\begin{enumerate}
\item Fewer puzzles. The puzzles in Zelda are essential and awesome, but sometimes they are overdone.  
\item Less formulaic boss battles.  Sometimes the boss battles feel exciting and heart-pounding, but once a player learns the formula for winning, the battle feels like a box-checking exercise, albeit with some execution risk.
\item More skill-oriented growth for Link.   I'd tip the game toward an advancement system where Link learns more skills as opposed to finding equipment that performs a certain function.  This already exists in some forms in the Zelda games, such as when Link learns new attacks in Twilight Princess, but I'd emphasize it more and make the learning essential for progression.  
\item I'd also like to bring ninja-related concepts into the Zelda universe, perhaps in a compartmentalized way like the Gorons or the Zoras.  
\end{enumerate}

To address these four points, I would do the following.

\begin{enumerate}
\item The major dungeons would be broken up by major quests which take place in the outer world.  The quests would not be major additions, since most Zelda games have them (finding the 8 pieces of triforce in Wind Waker, fighting your way through the woods to find the master sword in Twilight Princess, etc.).  The difference is in the ratio.  In addition to the higher quest/dungeon ratio, there would be more minor dungeons, which would have less involved puzzles.  
\item I have 2 methods for decreasing the formulaic nature of the bosses.  The first is to have humans play as bosses.  This solution seems infeasible, but one can dream, right?  A human player would dramatically increase the thrill and difficulty of a boss battle.  The 2nd method is to build more sophisticated and complex bosses.  For example, many of the bosses in The Legend of Zelda games are immune to every attack except for one (usually the attack is from the item collected in the dungeon of that boss).  This is a great and conceptually clean system, but I think there is room for a Mega-man style system where the boss is very weak against a particular attack, but not totally unaffected by others.  In addition, Link beats many of the bosses by first using an item to expose a weak point and then slashing the weak point with the sword (e.g., The Imprisoned in Skyward Sword: first knock it down and then slam the spike into its head).  It feels quite lame when the boss just sits, stunned, waiting for you to hit it's weak point.  Next, the bosses are not random enough.  They always attack and defend in the same sequence.  The hardest (and most fun) enemies in Zelda are slightly unpredictable.  It would be easy to scramble the sequence a little bit.  Maybe when it's time for a boss to spit fire at you, it spits between 2 and 8 times, chosen randomly.  You shouldn't be able to know in advance, even if you've played the boss before.  Finally, the bosses should attack more.  Most of them don't attack as often as smaller enemies encountered throughout the world, and those more aggressive enemies are oftentimes more fun to battle!  The key goal of all of this is to make the bosses more exciting, less predictable, and harder.  I want my heart to pound the entire battle, rather than just until I figure out some formula.
\item  The solution for the third point is to have Link learn skills, much like Samus learns skills from aliens in Super Metroid.  For example, in my ideal game, Link would learn of a village of ninjas located high in the treetops (see point 4) of the forest.  In order to get there, he would need to learn the ninja art of jumping high.  (REALLY high.)  Link would learn how to channel his chakra to perform a super high jump, and gain access to the tree-top village.  In this example, the jumping concept has two major advantages: it makes Link seem cooler, and it eliminates the necessity of building all sorts of special objects for Link to use his equipments on (like hookshot targets or grappling hook posts.) 
\item My idea here takes the form of a group of ninja-like humans who live throughout the world..  Link would learn from them in many ways, like in the example above.  
\end{enumerate}


\section{Story}
My proposed story: Link and Zelda are switched at birth, in order to protect Zelda.  Link is taken from a humble home somewhere and placed in the royal castle at Hyrule, and the opposite for Zelda.  Nobody knows where Zelda ended up.  This is all in response to a prophecy that the royal child would stop a great evil (Ganon).  Link trains  as a warrior from birth to protect himself but lives an otherwise carefree lifestyle.  Zelda lives a humble life of work, and neither knows of the switch.  

Initially, Ganon works through a network of agents in the politics and commerce of the peoples in the Zelda universe.  He has agents in all of the cities and towns to gather information and build power.  Ganon wishes to find and use the Triforce to subject the people to his will.  Eventually, later in the game, Ganon's network begins to gain power internally in many important cities.  When this happens, Ganon begins to build an army, and starts to conquer towns and cities throughout the world.  He searches mechanically for the triforce and the royal child who might stop him.

Throughout the game, Link learns of his history, of Ganon and his plans, and realizes that he needs Zelda to oppose him.  However, Link must find Zelda himself since nobody knows her whereabouts.  In the meantime, Zelda discovers that she is special.  

A character called Raf progresses alongside Link throughout the game.  Raf learns many of the same skills as Link, and begins the quest on Link's side (as a friend and servant of the royal family), but eventually betrays Hyrule by teaming up with Ganon, and informing him about Zelda.  Raf is the penultimate boss battle in the game, and brings to bear many of the same skills and equipment as Link.  Raf loses to Link and feels disgusted with himself.  Link refuses to kill him, and Raf leaves in shame.  Raf soon discovers that Ganon has used him as a pawn and plans to kill him.  So Raf later returns to aid Link in his fight against Ganon.

Zelda's role in stopping Ganon is as a receptacle for the Triforce.  She is, as the prophecy says ``the royal child of golden blood."  Zelda's blood can contain the power of the Triforce.  The Triforce exists in a separated form in the world, and Link collects the pieces and returns them to Zelda.  Ganon, however, has one of the three pieces, which Link must steal from him at some point.  Once Link collects the three pieces, and Zelda assimilates them into her blood, it becomes clear that Ganon had polluted his piece of the Triforce with evil magic.  When the 3 pieces are united inside of Zelda, she is poisoned by the polluted portion, and Link must cleanse her.  When he does, Zelda can imbue the Master Sword with the power of the Triforce, and Link can use it to defeat Ganon.

\section{Advancements}
Link will gain items, a hideout, a horse, new skills, and upgrades to each of the previous things as he advances throughout the game.  The following list names each advancement.

\begin{enumerate}
\item A Hideout, gained in Quest \#1
\item A horse, gained in Quest \#2
\item Speed boots, g
\end{enumerate}

\section{Outline}

This game outline includes 5 major dungeons, 9 (mandatory) quests, and 5 minor dungeons.  The major dungeons and quests would be designed to take 2 hours to complete, and the minor dungeons 1 hour, which means 33 hours in gameplay for these elements alone.  The other portions of the game (storyline, conversations, world-exploration, money gathering, etc.) would hopefully take the game to 80-90 hours of gameplay for an average player.  Of course, as with all Zelda games, optional mini-games and item hunting could take the gameplay time much further.

Here is my outline of the game.

\begin{enumerate}
\item Initial story about Link and Zelda being switched, as a narration.
\item The game opens as Link learns of evil in the political system of Hyrule, and then accidentally discovers that he does not have royal blood and that he was switched at birth with a child who lives somewhere in the world.  
\item A sage of ancient times, one of the authors of the prophecy about Zelda, appears to Link and charges him with the task of finding the true child of prophecy.  The sage gives Link a magical ring through which the sage can speak to Link.  The sage will serve as the guide for the game, much like Navi or Fi.
\item Link confides in Raf about what he has learned.  At roughly the same time, an agent of Ganon takes control of Hyrule.  After some investigation, Raf discovers that Link may be in danger of being killed by the agent of Ganon who is now in charge.  They decide that Link should leave the castle.  Raf will stay behind to gather information and keep Link informed.  Link changes out of his royal clothing and leaves the castle to obey the sage who told him to find Zelda.  Raf and Link devise a communication system where Raf will leave letters in a place in the castle courtyard.  
\item \textbf{Quest \#1: Find a Hideout}  The sage will tell Link (via the ring) that he must find a home base while away from the castle.  First he must talk to many merchants in the city.  He will find one who needs a night watchman for a caravan.  Moblins have been attacking the caravan and taking the goods, and if Link can stop them for good, the merchant is willing to let Link live in a building he owns in the wilderness.  Link will stop the Moblins, but will find that the promised building is in fact the Moblin's hideout.  The merchant will explain that Link must rid the hideout of the Moblins.  This will require sneaking into the hideout when the majority of the Moblins are away for a raid.  Link will have to defeat the Moblins who are left behind, including the boss, and then lock the doors of the hideout before the Moblins get back.  Link will place the jewelry and weapons of the Moblin boss on the front gate, and the Moblins who return will find the doors locked and the remains of their boss,  believe the hideout is haunted and go elsewhere.  A cut scene will appear where Link sleeps for the remainder of the night.
\item \textbf{Quest \#2: Get a horse} In the morning the sage will tell Link to begin searching the world for Zelda.  The sage will explain that he should start by searching for her among the people of the grasslands around the hideout.  The area consists of many small towns, farms, and trade roads among rolling grassy hills.  In order to search more quickly, the sage will advise the use of a horse.  Link must find the horse   
\end{enumerate}



\end{document}  