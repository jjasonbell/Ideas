%%%%%%%%%%%%%%%%%%%%%%%%%%%%%%%%%%%%%%%%%
% Short Sectioned Assignment
% LaTeX Template
% Version 1.0 (5/5/12)
%
% This template has been downloaded from:
% http://www.LaTeXTemplates.com
%
% Original author:
% Frits Wenneker (http://www.howtotex.com)
%
% License:
% CC BY-NC-SA 3.0 (http://creativecommons.org/licenses/by-nc-sa/3.0/)
%
%%%%%%%%%%%%%%%%%%%%%%%%%%%%%%%%%%%%%%%%%

%----------------------------------------------------------------------------------------
%	PACKAGES AND OTHER DOCUMENT CONFIGURATIONS
%----------------------------------------------------------------------------------------

\documentclass[paper=a4, fontsize=11pt]{scrartcl} % A4 paper and 11pt font size

\usepackage[T1]{fontenc} % Use 8-bit encoding that has 256 glyphs
\usepackage{fourier} % Use the Adobe Utopia font for the document - comment this line to return to the LaTeX default
\usepackage[english]{babel} % English language/hyphenation
\usepackage{amsmath,amsfonts,amsthm} % Math packages
\usepackage{amscd}
\usepackage{amssymb}

\usepackage{color}
\definecolor{light-gray}{gray}{0.95}
\usepackage{natbib}

\usepackage{sectsty} % Allows customizing section commands
\allsectionsfont{\centering \normalfont\scshape} % Make all sections centered, the default font and small caps

\usepackage{graphicx} % For graphics

\usepackage{listings} % For displaying code
\usepackage{courier}
%\DeclareGraphicsRule{.tif}{png}{.png}{`convert #1 `dirname #1`/`basename #1 .tif`.png}
\lstset{
         basicstyle=\footnotesize\ttfamily, % font is monospace and small
         %numbers=left,              
         numberstyle=\tiny,          % size of numbers
         %stepnumber=2,               
         numbersep=5pt,              
         tabsize=2,                  
         extendedchars=false,         
         breaklines=true,            
         keywordstyle=\color{red},
         stringstyle=\color{white}\ttfamily, 
         showspaces=false,           % white spaces
         showtabs=false,             % no character when tab is pressed
         xleftmargin=25pt,
         framexleftmargin=17pt,
         framexrightmargin=5pt,
         framexbottommargin=4pt,
         %backgroundcolor=\color{lightgray},
         showstringspaces=false      % don't show spaces    
 }

\usepackage{fancyhdr} % Custom headers and footers
\pagestyle{fancyplain} % Makes all pages in the document conform to the custom headers and footers
\fancyhead{} % No page header - if you want one, create it in the same way as the footers below
\fancyfoot[L]{} % Empty left footer
\fancyfoot[C]{} % Empty center footer
\fancyfoot[R]{\thepage} % Page numbering for right footer
\renewcommand{\headrulewidth}{0pt} % Remove header underlines
\renewcommand{\footrulewidth}{0pt} % Remove footer underlines
\setlength{\headheight}{13.6pt} % Customize the height of the header

\numberwithin{equation}{section} % Number equations within sections (i.e. 1.1, 1.2, 2.1, 2.2 instead of 1, 2, 3, 4)
\numberwithin{figure}{section} % Number figures within sections (i.e. 1.1, 1.2, 2.1, 2.2 instead of 1, 2, 3, 4)
\numberwithin{table}{section} % Number tables within sections (i.e. 1.1, 1.2, 2.1, 2.2 instead of 1, 2, 3, 4)

\setlength\parindent{0pt} % Removes all indentation from paragraphs - comment this line for an assignment with lots of text

\DeclareOldFontCommand{\rm}{\normalfont\rmfamily}{\mathrm}
\DeclareOldFontCommand{\sf}{\normalfont\sffamily}{\mathsf}
\DeclareOldFontCommand{\tt}{\normalfont\ttfamily}{\mathtt}
\DeclareOldFontCommand{\bf}{\normalfont\bfseries}{\mathbf}
\DeclareOldFontCommand{\it}{\normalfont\itshape}{\mathit}
\DeclareOldFontCommand{\sl}{\normalfont\slshape}{\@nomath\sl}
\DeclareOldFontCommand{\sc}{\normalfont\scshape}{\@nomath\sc}
\DeclareRobustCommand*\cal{\@fontswitch\relax\mathcal}
\DeclareRobustCommand*\mit{\@fontswitch\relax\mathnormal}


%----------------------------------------------------------------------------------------
%	TITLE SECTION
%----------------------------------------------------------------------------------------

\newcommand{\horrule}[1]{\rule{\linewidth}{#1}} % Create horizontal rule command with 1 argument of height
%the below reassigns the function \texttt{} the name "code".  So instead of typing \texttt{blah blah} you can %type \code{blah blah}.  It's just easier.

\newcommand{\code}[1]{\texttt{#1}}

\title{	
\normalfont \normalsize 
%\textsc{The University of Iowa} \\ [25pt] % Your university, school and/or department name(s)
\horrule{0.5pt} \\[0.4cm] % Thin top horizontal rule
\huge Idea Repository \\ % The assignment title
\horrule{2pt} \\[0.5cm] % Thick bottom horizontal rule
}

\author{J. Jason Bell} % Your name

\date{\normalsize\today} % Today's date or a custom date

\begin{document}

\maketitle % Print the title

%----------------------------------------------------------------------------------------
%	Introduction 
%----------------------------------------------------------------------------------------

This document is composed as much of aspirations as it is of ideas.  I dream of accomplishing three things in my temporal work:

\begin{enumerate}
\item  Becoming a specialist in something with which I can provide value to business.  My expertise must be capable of providing insights that improve net profits.
\item Consulting for at least one business
\item Creating a product of value.  Songs, a website, or a tangible good all would work.
\end{enumerate}

These three goals guide the following two lists.  The first section is a list for holding ideas about research, or just interesting ideas.  Hopefully, this list will bear fruit in the form of projects that advance my career.\footnote{There are several strategies for generating research ideas.  
\begin{enumerate}
\item Read other research
\item Let it happen by accident
\item Solo pondering
\item Many mechanistic approaches.  For example, maybe I should just think of data sets that are really boring, like John Rust did with bus engines.  Then sell the insights.  
\end{enumerate}
}  The second list is more general, but aimed at business.   


%----------------------------------------------------------------------------------------
%	Research List 
%----------------------------------------------------------------------------------------

\section{Research}

\begin{enumerate}

\item \textbf{Product Color Options} What if you studied a product design choice as simple as how many and which colors to offer.  How big of a difference could it make?  What about the black macbook as a case study?  You give a different color, up the price, and you it is considered an exclusive color.  Pretty awesome.

\item \textbf{Magnitude Studies.}  Someone could get famous for cataloging tons of important and famous effects and doing the magnitude in more realistic contexts.  Like, yes, all these CB effects exist, but are they big enough to change decisions?  Statistical versus practical significance. 

\item \textbf{Imitative Behavior of Firms.}  For dynamic policy functions, maybe I just do what you do, so value functions aren't unique.  That is, they may tend to converge by imitation.  Networks and dynamics come together.  

\item \textbf{The Unexpected Value of Platforms}.  Come up with some valuation mechanism and show that the error term is huge?

\item \textbf{The Value of Graphic Design.}  Check national trade associations of designers for these data.  If it doesn't exist, maybe an enterprising person (me!) could collect it.  Find local designers, and clients, and measure revenue, etc. before and afterward.  

\item \textbf{Adjacent Complementarity Across Goods}.  It seems that having more stuff introduces more desire for stuff, because of complementarity.  If you don't have a house, you don't need a couch.  If you only own one car, you only need to fuel one car.  If you don't own an iPod, you probably don't buy music.  

\item \textbf{The Value of a Rival.}  Is it possible that having a very strong rival actually keeps competitors out?  I revise my probability of success downward because I see the extremely fierce competition between Coca-Cola and Pepsi.  Also between Samsung and Apple.  

\item \textbf{Feature competition.}  Use Sudhir as a pattern and model the auto industry, maybe the luxury market.  The model takes into account introduction of new features and the copying of them. 

\item \textbf{A large scale market simulator.}  You basically build a building, with computers, allow people to learn the rules of the market for a period of time, maybe a week or a month, and then have them engage in market activity, perhaps with real payoffs.  Then you basically reserve time on the market simulator.  You'd use business students as subjects primarily, but anyone who wanted to could use it.

\item \textbf{Visual Brand Spillovers}  Do the brands that are closest to the luxury brands on the shelf seem more or less luxurious?

\item \textbf{Novelty Premiums.}  Newly released golf clubs have huge premiums but don't really have technological improvements to account for the price difference.  What psych/econ model of choice can capture this?  
   
\item \textbf{Induced complementarity.}  If you try harder on this one, maybe you can do it.  A system that doesn't have to be a system, but can be? 

\item \textbf{Scale Attraction in Unit Sizes.}  Laundry detergent producers like to put higher concentrations in each unit, (2x concentrated).  It takes up less shelf space and you use it faster.  It might be nice to estimate the magnitude of usage rate increase using a structural model.  Say, if I'm thinking of trying to boost usage rate with a higher concentration of something, will it be worth it?  

\item \textbf{Exclusivity in Online Groups.}  Gmail and Birchbox both require users to apply, or be invited by current users in order to use the service.  In a market with network effects, how do companies balance the desire for more customers and the desire to maintain the perception of exclusivity.  In general, exclusivity seems to come from price, but there are some things that are exclusive independently of price, like being the first to see a movie.  This is an interesting subject to me.  It relates to Veblen goods.  It is also related to the quality distribution in the iTunes store vs. Android store.  Android lets everyone in, iTunes filters.  It could be as a function of the prices charged to developers?  Or it could be a screening process.  Screening can improve quality \textit{and} enhance exclusivity perceptions. 

\item \textbf{Product line pruning.}  Ford removed the Ranger from their lineup, in hopes that buyers would move to the F-150.  Is that true?  Myatt and Johnson have a paper about product line pruning that might be applicable. 

\item \textbf{Feature pruning.}  What things can you profitably remove from a car?  This is related to the literature on SKU deletion.  Apparently the removal of redundant items is very good for sales (contrary to what Manzi claims.) (Boatwright and Nunes, 2001).  However, these SKU's were specifically chosen because they are low selling.  The take-away seems to be that it is difficult to remove SKU's but it can possibly increase sales if you do it correctly.  There are a few items to consider in this context. 
	\begin{enumerate}
	\item Overall sales may increase, but what about alienating the few customers who really cared about the feature
	\item Boatwright and Nunes (BN hereafter) consider variety.  The product analog of variety is functionality.  The product analog of ease of decision is usability.   
	\end{enumerate}

\item \textbf{Modeling an ``x-factor."}  This can't just be any error term, it has to sometimes be really big and sometimes be really small.  Tri-modal, maybe?  A mixture of normals?

\item \textbf{Kuhn and Christensen.}  Thomas Kuhn and Clayton Christensen both got major leverage from basically the same idea: progress is mostly incremental, but is interrupted by jumps.  Kuhn's jumps are paradigm shifts, and Christensen's are disruptive technologies.  Perhaps I can apply this idea more deeply in a modeling format.  Basically I can model disruptive innovation.  I'll have to see.  The Lucas Critique is that models don't capture parameters that are deep enough to survive policy changes.  Perhaps even current structural models aren't deep enough to capture changes in innovation.  Could an economic model have predicted the massive adoption rate of tablets?  Probably not.  The models don't provide much understanding of why people have adopted them, even now.  It seems that we must incorporate psychology to truly get it.  

%\item \textbf{Innovation Strategy Comparison}  Two strategies. The first is to face uncertainty about engineering constraints but knowing the market wants a particular product.  The other is to use proven engineering tech, but compromise what you know the market wants a little bit.  Which one wins under which circumstances?  Which one do people use?

%\item \textbf{Catolog of Psych Phenomenon}  Basically a dictionary.  You could compile it from Wikipedia.

\item \textbf{Aggregate Lattitude of Acceptance}  If I could find out when it exists, I could find instant 1\% windfalls.  It must exist in the aggregate though. 

\item \textbf{Product Change Responses to Entry} Lots of price response lit, not much product, even though John Murry would say that's the way to go.

\item \textbf{Pricing Strategy Competition}  Are there circumstances where cost-plus is a best response?  Related to the optimal research.  

\item \textbf{Squatting} your way into the market vs. advertising your way into the market.  I just have to be the only guy around who does X and I can beat a lot of advertising

\item \textbf{Package Sizing Attraction Effects} Read Sanghak's paper about indivisible demand.  The idea here would be to see what sizes firms should make their products to maximize profit, especially fast food firms.  Should you let adults order off the kids' menu?

\item \textbf{Assortment Offerings and Customization}  Where are the industries that might benefit from customization?  Industries that can make the custom part modular (like hemming mass-produced pants), industries with ``alignable'' attributes.  A lot of the problem comes form needing to figure out which size to offer, so make it truly continuous.  Interesting.  You could compare purchases at self-serve places to purchases of the same item when assortments are pre-selected by the producer.  Like all-you-can-eat buffets vs. made-to-order restaurants.  

\item \textbf{Product Vacuums}  You could have different theories about entry into the product markets.  Then you could evaluate each empirically and try to come up with a framework for new product development.  For example, suppose you have a characteristic space and you find pockets that balance emptiness with nearby successful products.  Use other products as a way of `filtering.'  That way maybe you can find arbitrage opportunities: holes in otherwise rich markets have good potential.  If a firm used this entry strategy, maybe it could make some money.

\item \textbf{Endogeneity Impact Tests} Use the BLP model with instruments to ``fix'' the fact that price is correlated with unobserved product attributes: then add a bunch of product attributes to see how it changes things.  You should be able to get to a point where you can say something about the use of the instruments.  

\item \textbf{Pricing Under Demand Uncertainty} Follow \cite{hitsch2006empirical} but instead of investigating new product launch, investigate pricing.  Maybe you can find out that it's optimal to do cost plus, or maybe you'll find, as Hitsch did, that simple decision rules lead to large profit losses.  Bayesian decision theory can act as a guide about how much you should learn about demand.  See \cite{hitsch2006empirical} for references and check to see who cited it, because this has probably been done a lot.

\item \textbf{Price Setting Policies}  Try to model equilibria where firms set prices in different ways: cost plus, competitor-based pricing, and demand-based pricing (the best kind).  Then you can go around and see what pricing evidence suggests is actually happening.  This would be similar to \cite{delossantos} did with search paradigms. 

\item \textbf{Machine Learning and Marketing}  Machine learning methods often use iterative procedures based upon a series of regressions.  If at some point an endogenous variable is entered into the model though, the residual is an inconsistent estimator of the true error and so you can contaminate future estimates.  It is hard to simply guess about this issue though.  Perhaps machine learning methods don't suffer as much from endogeneity.  
	If you can solve a structural model to get  an estimable reduced form, machine learning may have advantages.  However, the reduced form dictates which variables enter the model, and so in the end you may ``throw away'' the value of the machine learning.  
	Counterfactuals are also difficult to think about.  I can predict what would happen, short term, but I'm not sure about longer terms, and unstable situations.  Maybe if you have a training set and a test set with a markedly different environment, you could start to understand how machine learning performs.  Maybe I should compare my VCC model to a machine learning model.
	I should reread ``Uncontrolled.''

\item \textbf{Implications of Price Unawareness} Affluent shoppers don't know the prices of things very much, poor people know very well.  Also, relative prices matter a lot.  The average price of the category should be a big indicator.  Context effects.  Also, expertise can reduce dependence on relative indicators.  

\item \textbf{System 1, System 2 Utility Function} What would it look like?  Could you estimate weights on it?  That could be very cool.

\item \textbf{Deiter Rams Rules of Design} Translate them into math and find out if he's right.  Do it the same spirit as the ``Gut Liking for the Ordinary'' paper.

\item \textbf{Share vs. Profit} Say you have a temporary product advantage.  What will be better: charging a premium until competitors catch up, or acquiring more market share?  Which one will have a better lasting impact?  This could be a big question.

\item \textbf{Dynamic Pricing and Planned Obsolescence}  If a manufacturer can make the product obsolete by design, maybe it changes how they can price.  It dampens competition between generations of products.

\item \textbf{Perceptual Mappings and Physical Attributes}  There are more than one way to get to the same spot in perceptual space.  Not just more than one set of attributes (many-to-one mapping), but also advertising can create perceptions (this brand is perceived as high-quality even though it isn't), and your brand identity can have this influence.  So can price.  What are the most important perceptions to have?  What is the cheapest way to get there?  

\item \textbf{New Forms of Price Discrimination}  You can price discriminate using many things: time (Airline seats, new products), any segmentation variable (age, student or not, fast or slow users), product (size, quality level, format), distribution channel (high-end vs. low-end stores).  It might be interesting from a practitioner's point of view to search the space of variables on which you can price discriminate and then pick the `best' one.  Questions to answer: is price discrimination feasible?  Will it get me a boost in profits?  Will people get mad?   

\item \textbf{Sneaky Price Increases}  Some brands may be able to raise prices without anyone noticing!  Others may be much more committed to prices because of media or their business model: Netflix is doubly committed, because everyone pays a monthly fee and the media will talk about it so much.  Campbell's soup?  Nobody will notice if they just charge 1\% more.

\item \textbf{Tradeoffs Between Certain, Risky, and Uncertain Attributes} Suppose a decision maker sees a car and knows that a) he likes the styling, b) he has a strong idea of the reliability of the car, and c) knows the safety rating but doesn't know his likelihood of getting in an accident.  The first attribute he is certain about, the second he treats as a risky attribute, and the third he doesn't even know the risks.  How will he trade off between these attributes?  He can't even use expected utility on the third one!  Maximin for one attribute, constant for another, expected utility for another?

\item \textbf{Partially Two-Sided Markets} The chicken-and-egg problem arises in two-sided markets: you need side A to entice side B, but you also need side B to entice side A.  Apple solved the two-sided market for smartphone apps accidentally: they got people to like the phone for its own sake.  Then they built a two-sided market on top of their existing user base.  You already have side A all filled out, and the two-sided market works.  Mind Your Major could pull the same strategy.

\item \textbf{Decreasing Marginal Utility of Money}  Under some circumstances it may make sense that money has diminishing marginal utility.  Those circumstances are immediate cash infusions.  My immediate utility of \$100 is probably less than 100 $\times$ my utility of \$1.  However, this is only in the short term.  Eventually you habituate.  I guess the real distinction is short-term vs. long-term.  

\end{enumerate}


%----------------------------------------------------------------------------------------
%	General List
%----------------------------------------------------------------------------------------

\section{Value Creation}

These are business ideas, or ideas that I would love to implement one day, or inventions, or just thoughts or observations.   

\begin{enumerate}

\item \textbf{Analog/Digital Notebook} It's paper, so it feels like writing on paper.  As you write, the notes digitize themselves, and then the ink disappears after a while (maybe on command?).  So you have a single notebook forever, and all the benefits of digitization.  Kind of cool.  There are lots of products that would be awesome if you could combine analog interfaces with digital storage and flexibility.  

\item \textbf{Beta Crowd} You sign up a bunch of people who use a product and then predict whether it will fail or not.  Check success rate.

\item \textbf{Passive Conjoint} I couldn't figure out a better term, but the idea is that you could have a website collect conjoint data by modifying prices, etc.  Or by watching ebay postings or something like that.  The idea is to use existing methods to passively collect conjoint data in some way.  There has to be a way.  Customization?  Not sure.

\item \textbf{Product Vacuums}  Try to define metrics for where things have a lot of added value.  Could you find a metric that measures how intensely design has been applied in a market?  For example: fashion, yes.  Dental floss, not so much.  Maybe it's hard with floss, so you'd have to come up with a difficulty metric too if possible.  This seems hard, but something to think about.

\item \textbf{New Education Delivery Systems}  Startups which improve MOOC retention rates have a good chance of profiting.  Startups which improve the communications technology have a good chance too.  Maybe like the way the Church does it for Priesthood meetings?  You have lectures that are live but broadcast them to buildings?  This is much like the Yale class I took on game theory, except I didn't get credit so I dropped off.  

Another idea is targeted skill-based mini-classes.  Like, you need to learn to a very specific technical skill, so you pay an engineer to teach you (it's got to be big money to make it worth his while.)  The basic problem to be solved is that some people are trying to get a degree, and others are trying to learn a skill.  Different approaches are necessary for the two groups.  The ones who need a skill don't want to take an entire course.  YouTube partially solves this problem, but there is a long way to go.  I'll bet most specialized skills are not worth sharing on YouTube.

You could have people trade their skills into the network in return for the right to view other courses.  I get credits for people watching my video.  If you created a market place for 1-on-1 skills sharing you might have it made.  It's a platform and people will extract the best value from it.

You could also structure it like a ``knowledge brokerage.''  We setup course plans for people who want to achieve some certain thing.  Maybe a coursera here, an edX there, an online certification, some on-the-job training and boom you have what you need.  You'd need to have relationships with businesses, but maybe businesses can get a cut from providing training.  (Seriously, some businesses ought to be accredited.)  

\item \textbf{MTurk Conjoint + TRIZ}  You can run conjoint on MTurk.  If you could automate the process, you might be able to close the loop in automating innovation.  TRIZ trends or Patent mining to generate ideas, MTurk to evaluate them.  

\item \textbf{Urine = Ferilzer?}  You pee in a hydroponics system.  Saves water, grows plants.  Bam.

\item \textbf{Meta, Micro/Macro Gaming.}  You play the villain in Zelda against another human who is playing as Link.  Harder, more realistic battles for Link, with tons of replay value.

\item \textbf{Color Changing Conductive Fabric} Buy shoes and download color palettes for a dollar.

\item \textbf{Idea Blender} A website with tons of concepts that can be randomly matched.  Maybe not randomly later (so you have a problem and a list of similar problems with their solutions.) 

\item \textbf{Ethnographic Videos.}  An ethnography is a tool used for creating new products.  For example, Swiffer is the result of an ethnography, along with many other well known and successful products.  The basic idea is that you watch consumers while they use a product to try and see things in their behavior that aren't easily verbalized.  This technique is very powerful, and essentially unused by small businesses.  We could try and target small businesses. \textit{Growth Hypothesis:}  Word-of-mouth, and perhaps repeat use (for multi-product small businesses.) \textit{Value Hypothesis:}  This is tough.  Small businesses may not be able to see the power of this technique.  We may need to do a lot of convincing in the beginning.  It is likely that we only need to get a few clients in order to succeed, since the technique has a very high success rate.  

\item \textbf{A Media Recommendation Service.}  There are many recommendation services that rely on algorithms.  These are designed for a broad market.  What I have in mind is a niche market.  We provide extremely high quality, customized recommendations for customers.  I can see this working for books, music, movies, and TV shows.  We could start out by getting a single client and giving them a concierge style service in order to learn more about the growth and value prospects of the idea.  As an example, imagine an employee of Goldman Sachs in SLC, say John.  He is busy with work all of the time, but is making quite a lot of money.  He likes Pandora but the recommendations leave much to be desired, so he hires us.  We find out which bands he likes, find something new for him, and as part of the price we gift the songs to his iTunes account.  Maybe it could be set up as a subscription, or maybe as a three month consultation where we recommend, get feedback, recommend again, feedback again, and so forth until we unlock a large body of suitable media.  Essentially, we would be personal curators for pretty well off people. \textit{Growth Hypothesis:}  I think this would spread by word of mouth.  \textit{Value Hypothesis:}  I find myself often searching for new movies, music, or TV shows that I like and are clean.  I think recommendations from Amazon, Pandora, etc. are okay, but a person could do a lot better.  I think there are more people like me.

\item \textbf{Living Video Game.}  We'd manufacture quests for people.  They'd get a team together, pay an entrance fee, and go through a series of challenges to eliminate other teams.  The prize would be something awesome.  An example: we get a lockbox, fill it with \$100 dollar bills, and hide it somewhere.  Teams who enter the contest all get keys to the lockbox.  The first team to find it gets the money.  On the way, we give out clues to each team, or maybe teams would have to enter challenges to get clues.  We could design it so that teams entering a challenge would have to play against each other in order to get clues (and losing teams are taken out of the running).  It's a lot like a game show, but since you pay to be on, the supply is unrestricted.  We could do all sorts of stuff: paintball fights in the forest, riddles and clues hidden underground or in weird places in the city, interacting with actors hired by us to get information, espionage between teams, etc.  It would be awesome.  The biggest question is time frame (since people are busy), but we could figure that out pretty easily I think.  \textit{Growth Hypothesis:}  Paid growth, word-of-mouth.  Since people would need to get teams together, it would be forced to grow.  We could allow any number of people to form a team, but they'd have to split the winnings evenly, so less incentive to make a huge team.  \textit{Value Hypothesis:}  People like adventure and escaping from everyday stuff.  People like to think of themselves as competent, and they like challenges.  Plus, everyone wants to open a lockbox full of money and get to take it home.  

\item \textbf{Open Source Style}  This idea might be below.  You just post pictures and people give you style tips.  Whitney thinks it stinks, I like it. 

\item \textbf{Resizable Physical Display}  Maybe with projectors you could do this, but it would be awesome if you could resize physical screens.

\item \textbf{Thinner Mattress} Lots of engineering required.  Or maybe you could just get mats and stack them?  People who are willing to sleep on fewer mats have to pay less.  Just a new pricing model, I guess.

\item \textbf{Better Trash Can}  Can someone redesign the trash can?  Is there a better way?  Papers for sure need their own spot.  Maybe just a trash can that allows for easy recycling?  What about a trash can that you can put plastic into and it melts them into pellets?  A consumer recycling device!

\item \textbf{ The Contrapositive of Apple}  People pay a lot for Apple products because they are beautiful and solve problems.  Would they pay a lot if they could get rid of ugly things that cause problems?  Yes.  What are those things?  Maybe just removing them would be good.  A service?  You could just give a free consultation to people.  You go to their house, and tell them exactly what to do and charge 5 bucks for it.  (Maybe, get rid of your fridge, for example.)  What if you just had a service that came into your house and emptied your trash?  How about this: a service where neighbors could ask someone to remove ugly stuff from the yard.  None of the neighbors would be implicated.

\item \textbf{A Real `Spaces' Desk}  A physical desk that allowed you to switch workspaces quickly.  Maybe expose as well? 

\item \textbf{Problem, Meet Research}  A service that people could bring their problems to and be acquainted with relevant research in the field in a very readable format.  Customizable journalism, I guess.

\item \textbf{Databox}  Like a video game console, but only for exploring data.  You load the data in using wi-fi or the web or something, and then you can explore it like crazy with lots of visualizations, etc.  

\item \textbf{A Better Analog Calculator}  Touching two numbers at a time would mean add, for example; or maybe you would draw the operator on the second button.  So, touching 2 and then a horizontal swipe on 1 returns 1.  How about just a huge calculator.  You could build it into a wall.  

\item \textbf{Jean Fit Box}  You create a huge number of jean fit types and send people a box with them all.  They try them on and get their perfect size and then they can get custom jeans forever.  What about something they use more often?

\item \textbf{Caring for the elderly.} is an industry that everyone instinctively avoids I think.  I'm pretty sure there are a lot of opportunities there that people just don't want to deal with.  I'm not sure what, however.  Can we get elderly people to do some light work?  It would be good for them and good for society.

\item \textbf{Bite Flosser.}  A device, shaped like a pair of nested U's, and with lots of floss running between the U's.  You bite down on it, so that each space gets a piece of floss, and then you move it forward and backward to simultaneously floss all of your teeth.

\item \textbf{Quirky for Songs.}  People like me have a few half-way formed songs, with melodies, lyrics, etc.  These users would submit their songs in any form they wanted, (most likely performing them on video).  The best songs would be fully created by a professional band.  Song submitters get a small royalty \textit{and} the satisfaction of creating a song, and bands get supplemental tunes; i.e. people do their work for them.

\item \textbf{Tailoring Service.}  Most clothes don't fit people very well, but they don't know it.  We'd set up a service, locally first, where customers could send all of their clothes and get custom alterations.  Lots of people search for a long time to find the right fit, but this way, they could buy from almost anywhere.  The nice thing about it is that your measurements would be saved in the service so you could reliably get well fitting stuff.  This type of service already exists, but it mostly deals with newly fabricated items, which are very pricey.  This takes advantage of economies of scale in the textile industry, but achieves results that are nearly as good as custom.  We'd also face a lot of local competition, but I think we could handle it.

\item \textbf{Favor Network.}  This one needs more thought, but it would probably have to be a Facebook app.  People would ask for favors, and help others with favors.  Helping someone with a favor would let you build credits of some kind, and then you could pay for other favors with those credits.  You could also buy credits with money.  An example favors: borrowing a textbook, getting a ride from A to B, using someone's parking spot, an introduction to someone in order to date them, help organizing an apartment before cleaning checks.  The asker would set the price, so the value of doing a favor would be determined by a market, which would save us a ton of work.  All credits would be redeemable for favors, or for money (but if you bought Y bucks worth of credits, you could only sell them back for X dollars, where X < Y, which is how this thing would make money.)  There are networks somewhat like this, but none have broken into the public consciousness yet.  I think the best way to do so would be to focus on one type of task, like introductions for dates or loaning textbooks, much like AirBnB is a specialized network of this kind for allowing people to sleep at your house while you're away.  I believe keeping it tied to Facebook is a good way to catch on quickly.  Once we get into public awareness, the types of favors could be more general.  


\item \textbf{Consumer Storm}.  ``Hell hath no fury like a consumer storm.''  This can already happen, but maybe a formalized way of really hurting a company's reputation for doing something wrong or mean.  

\item \textbf{Software $\rightarrow$ Hardware}.  This is the opposite of ``software eating the world.''  You take something that exists in the world of software and try to replicate it in hardware.  Nimbus from Quirky, or 3D printing are examples.  
\item \textbf{Hi-Plug.}  A power strip that stands up, like a lamp.  It brings plugs to a reachable height.  The design is really critical, because lots of wires get ugly.  What does this have over an extension cord?  Maybe, it could be modeled after a lamp.  People have solved many problems here. 

\item \textbf{Crowdsourcing.}  What else could it accomplish?
	\begin{enumerate}
	\item Time, maybe?  You can beg for time and if you get it all you can clean up a park, move, or something?
	\end{enumerate}

\item \textbf{Automatic Furniture Rearrangement}.  If it were possible to have a system which could automatically rearrange your furniture, it would be extremely valuable.  Different configurations for: more space, games, eating, media, etc.  You could make one room much more functional and save so much space in the house.  

\item \textbf{The most reasonable diet in the world}.  You just do one very easy thing, like eat only one sweet thing a day, and then hope for modest results.  The same could be true of money, work, etc. 

\item \textbf{A springform muffin tin}.  If this doesn't already exist, it needs to.  It's perfect for Quirky.  Silicone ones exist but Whitney hates them.  Maybe there is an opportunity.

\item \textbf{People like to customize} what they use.  Worms and excite bike are two awesome examples of this.  Where else could you apply it?

\item Nobody pays any attention to \textbf{cardboard boxes}.  People care about plastic bags quite a lot, but not the boxes.  If there were a better way to handle it, I think there is a lot of money to be made.

\item People want to participate in \textbf{close-knit groups.}  They love to work together on projects they care about.  They love to create things and connect more deeply than on Facebook.  Networks are important for value.  

\item \textbf{Group Financing.}  If you could get friends to pool their money, you could buy things with zero interest.  Then, the website would make sure you get the money back.  This is a lot like lending tree.  

\item A service that \textbf{applies for hundreds of jobs} for a single applicant.  You give the resume, they find positions in your industry, apply, and you can schedule interviews with them etc.  

\item An \textbf{app that tracks new albums} from bands you like and designate.

\item \textbf{Entrepreneurial Toolbox.}  From lofty idea, across the valley of idea death, to a successful product launch, this toolbox provides data, consulting, and financial resources for entrepreneurs.  Y-Combinator for the masses.  

\item \textbf{Flexible Book Shelf}.  Each shelf in the middle could be flexible in some way, so large items could fit, and you don't need to rearrange the shelves?  Not sure how much value this adds.  It does solve the problem of having to fit all your tall books in one shelf, and then wasting tons of space.  Maybe there is a better way.   

\item \textbf{Artistic Watch Faces}.  A display made of flipping tiles, a display made of small rectangles that form a sun.  The sun grows as the day goes to midday and then shrinks back down toward night.  The main draw for these displays is beauty, like the visualizers in iTunes.  A new, novel way of displaying time that sacrifices accuracy for aesthetics.  

\item \textbf{Minimalist, Self-Cleaning Microwave}.  Bad thing about microwaves: they are messy looking in and out.  Pare down the interface and build in a nice way of self-cleaning.  Perhaps a compartment in the front that you pour cleaning liquid in and then hit a side button which turns on the microwave and pumps out the liquid so it evaporates.  Then, have some automated wipers or a tool that comes with the unit to very easily and quickly clean the sides, which should be easy since there is a nice layer of cleaner all over. 

\item \textbf{A General Purpose Tracking App.}  Examples: track when you buy produce so you can make sure you eat them before expiration.  Track when you purchase clothes and from where and how fast they become worn out.  

\item A catalog, perhaps called \textbf{Catologue 1}, which is a curated group of ultra-high quality products.  Like Michelen ratings but for consumer products.

\item \textbf{Workline}.  A website that coordinates several Etsy-style projects in a complementary way to create more value.

\item \textbf{A hollow couch} made with a very strong frame (carbon fiber or aluminum?) and one that you can open the back of for storage, cleaning, or because it's against a wall.  

\item  \textbf{Better Backpack}.  What about a backpack which moves more weight to the front of the body.  Backpacks are heavy but the weight could be distributed better by having the straps be much wider and cover more surface area.  Like a vest.  The front could even have large pockets so you carried more weight at the front.  What if you could leverage that fact that your legs are really strong to offset some of the weight carried by the shoulders when you move your legs?  Ivar backpacks got a patent for a shelving system inside the backpack.  It distributes the weight better by moving some up higher.  On 7/22/2013 I safety pinned an iPad case to the straps on my backpack and it worked nicely.  I also discovered that moving some weight to the front is really helpful.  Also, a backpack that weights its contents.

\item \textbf{MBA Student Labor}.  I like this idea: you take the consulting projects that MBA's do in classes, and do a meta-analysis-style quantitative aggregation of the results.  Then, you see if the aggregate does a good job of producing strategy.  If so, you could decrease the sample size and see how many people you'd need to produce it.  The real problem is this: you can only look at past ones.  You'd basically need a controlled experiment.  That might be difficult.  Maybe you could get some input from DJ and John Murry and maybe get one of those two or Tom Gruca to help you out collecting data by using their classes?

This idea can be nested in a broader one: Quantification of qualitative analysis.  For example, you could select a group of people and ask them to give ideas about something.  Then use word recognition and that strange fuzzy logic stuff to create a neural network (like the guy who predicts stock prices) and predict how the business should do.  Data mining has been criticized because of "garbage in, garbage out" so how about take a guiding theory and priming people.  Or, you could maybe use a bayesian prior to represent theory's contribution to the prediction. 

\item  \textbf{Shoes that "sweat"} on the bottom so you never lost grip

\item \textbf{\$ for a Meeting.}  A website where you could trade money in return for a meeting with the right person.  Example transaction: 
\$10,000 in return for an hour with the regional buyer of Wal-mart.

\item \textbf{Research database} that forced you to tag your paper with more than just keywords.  You'd have to categorize the topic, effect size, and whether you supported or didn't support a theory.  Then you could automate theory building, but suggesting theories and eliminating them using the database.  Maybe just searching the database for keywords would work?  I'm not sure....

\item \textbf{Avenue Sampling}.  You have a lot of possible avenues for research, theory, and methodology (I 
do personally so do businesses.)  Why not construct a distribution over them
and sample from each one.  Then you'll slowly trace out all the possibilities.  As
you iteratively experiment, you can find the strategies.  Hmmm...this is very
similar to "Uncontrolled."

\item An \textbf{oven that ``watches" your food} to figure out when it is time to turn off.  It could detect color shades like ``golden brown."  It would have to turn off early because stuff keeps cooking.  Maybe it could open vents and fan the air out so you wouldn't have to be there.  Similarly, a fridge that tracks when you put produce inside, and tells you when they expire.  Update: chips exist that can track food items.  Maybe the 'done cooking' concept works best in a toaster, actually.

\item \textbf{Household utility API's.}  An smartphone app paired with hardware sensors, placed around the house, to measure electricity usage, water usage, etc.  Like a bunch of API's linked to the phone.

\item \textbf{Better Skateboard Lighting}

\item \textbf{Better Transitional Sunglasses}  With a manual control, and force it to be completely binary.  No in-between stuff.

\item \textbf{Outfit Chooser}  Based on pictures it chooses which outfit to pick.

\item \textbf{Hand Me Downs}  You sign up for someone who is your size and style and spends a lot of money on clothes.  Then they have a guaranteed buyer and you have a guaranteed seller.  Could even do this with cars, or anything for that matter.  Early adopters get subsidized by laggards in return for taking the risk, laggards get low prices.

\item \textbf{Big Data for Little Businesses}  Sign up for a big data feed and get recommendations of some sort.  Maybe set up the kind of research facilities used by large companies and give lots of little businesses access.  Like, for example, execute the new product forecasting algorithm and then the results to little businesses.

\item \textbf{Exploding Scope}  Host gatherings with smart people across different industries.  Draw from creativity research to cultivate an environment for solving big problems.  Maybe people explain how they solved recent problems and try to apply across industries?  I'm not sure.

\item \textbf{Super Training} Is it possible to bypass training requirements and just have smart people start doing work?  Maybe you could just take high-school students and put them directly into an engineering job that would usually require a bachelor's degree and see what happened.  Pay them \$60,000 and they'd do it.  

\item \textbf{Positive Deviance Feed}  I guess the point of positive deviance is that knowing isn't enough.  Still, maybe you could find a way to simply have a data feed where you could see what the best people in your area are doing.  (Maybe they could be those who aren't directly competing against you.)  So, if you're studying for the ACT or SAT, you enter in a host of information about yourself, and both serve as a data point as well as see data from others.  Then, next generations can see who did the best.  Same principle with small businesses, phd students, engineers, product managers, etc.

\item \textbf{A Retail Store that Only Carries New Products}  Like Pre-Fundia for big players.

\item \textbf{Distribution of New Technology} We are just a catalyst for the spread of tech.  

\item \textbf{Curation}  Maybe experts post, and then people can veto.  Items with the fewest vetoes stay (max 3?).  A few material categories, a few experience, a few media, maybe a few in human capital.  

\item \textbf{Startup Mergers}  Why don't mergers happen at the startup level?  Maybe the Thomas Edison approach could work.  Start a battery company, merge with others, buy others, etc. until you start to get some serious scale.  Maybe 18 minimalist blogging platforms don't need to exist.  Maybe you could just buy up several of them and combine the human capital? 

\item \textbf{New Ideas from Crowds}  Crowds are really good at populating lists.  You find ingredients for good ideas, like ``technology that hasn't been updated for a long time," or ``important problems that haven't been solved," or ``things people care about but don't act on (the interest/action gap)."  Once you have such lists, you could assign groups of people to work on the different areas.  The current technology that exists to answer the question ``what market should I enter" basically consists of individuals thinking really hard.  Entrepreneurs don't have access to the kinds of information that large firms do.  A platform like this could change that.  
	
What items would be good to populate?
	\begin{enumerate}
	\item A clear definition of the problem.
	\item The skills necessary to solve the problem.
	\item The estimated value of solving the problem.  
	\item The other groups working on the problem, or the next best solution.
	\end{enumerate}

\item \textbf{Coordination Networks} Like everyone at  University is leaving and it would be great if we could all let some central system know so it could turn off lights and computers and save money.  Except, as of now we just have to leave it to each person.  You could put a nest in every room but it would be insanely expensive. 

\item \textbf{Skill Classes}  The Skills Gap is coming, so maybe plumbing, carpentry, and other types of classes will be of use for the Millenials who are very educated and have no skills.  

\item \textbf{Healthcare Fraud Private Investigators} We find healthcare fraud and take a cut.  For every \$1, the return is \$7 to the U.S. gov, so maybe there is a deal to be made.

\item \textbf{Web RA} You are a researcher and need someone to collect, clean, describe, or plot some data.  You pay Web RA to do this stuff.  They build up expertise and eventually automate most of the processes that are commonly requested.  Suddenly time consuming research is available to many more people.  

\item \textbf{Cause Manager} You help rich people prioritize their philanthropy.  They pay you to manage their giving.  You make sure it has the largest possible impact on the cause they care about.

\item \textbf{Rugs Warby-Parker Style} No middleman, made-to-order, super stain-resistant, Whitney designs the patterns.  Maybe modular rugs?

\item \textbf{Changing Color Necklace}  Glass in whatever shape and size you want, you change the color to match outfits.  

%\item \textbf{Emmoni-Nail Polish} You can do polish and emmoni nail at the same time.  This could work in a lot of ways: makeup and things that help skin.  It seems like an already well used idea, though.

\item \textbf{Electronic Heat "Breather"} When I put my mouth on my sleeve and breath, it gets really warm.  It might be a really efficient way to heat in a localized way.  Could you make a heater that did that?

\item \textbf{Responsive Design}  We mix qualitative, quantitative and experimental methods to understand a business context.  Joe does interviews, Logan films, Dan edits and prepares a video for the CEO, I run experiments and analyze any data, maybe estimate demand curves and check for pricing optimality.  Maybe we could hire designers, starting with Paul Heslop for creating new products.  The key to this model is that we give away consulting services in exchange for widely applicable insights into a domain.  We can also get contact information from people we work with.  An example scenario:
\begin{enumerate}
\item We ask a business that fabricates custom parts to let us film the process, talk to the employees, and run some numbers.  They let us do so in return for information: we can use whatever we learn to create products.  We can also enlist their workers to test the products.  In return, we offer consulting, a video of their employees at work while the boss is not their, interviews, a breakdown of how their employees are using the bulk of their time, any stand-out workers and what they do differently and if it can be applied elsewhere.
\item We have Joe do interviews of employees.  We have Logan go through and film.  We could even have an employee wear a mic and collect information from other employees secretly.  I would use the videos to document where people seem to spend the most time or have trouble.  I would match it up with Joe's interviews and see if we could identify the major problems.
\item We would give the business our results, but we would also try to develop new products to solve the problems we'd found.  In this case, maybe people need to wear gloves because the metal is so hot during some process.  We would find out how much money that costs and maybe invent a new method (process improvement style) or a new product that eliminates the need, etc.
\item We give hand over the tasks and information to designers and engineers and create a new product, and we sell it to businesses, starting with the one where we developed it.
\item Repeat.
\end{enumerate}

\item \textbf{Major Picker} Use open net from the government to help people pick majors.  It's a website so it scales and can take the entire education market by storm.  We compete with counselors, and we will crush them.  Our site will promise to help students pick a major in 2 weeks for 20 dollars.

\item \textbf{Pumpless Soap Reservoir}  Why do we need the pump?  You can just dip your finger in the soap and you get the right amount.  It saves packaging.  

\item \textbf{Liquid Networks} Steven Johnson's idea is that ideas come from slow hunches and liquid networks.  The entrepreneurial world seems to be obsessed with liquid networks of engineers.  What about liquid networks of something else though?  What about a liquid network of artists?  I but you could capitalize on undervalued resources by doing this on purpose with different networks.  Not very many startups get a big team of motivated botanists together, but that could be killer.  You could get even better ideas maybe by adding a few random people to the network, like botanists plus an engineer plus a finance person plus a designer.  

\item \textbf{Combinatorial Seeds} Netzer and Toubia figured out a way to incorporate humanity into an algorithm by using google.  Google represents the content of a particular domain, as created by humans.  There are possibly other ways to do the same thing that could be useful: instead of a population of webpages, it could be a population of journal articles, patents, recipes, or a single person's ideas.  I could do a combinatorial algorithm on my own ideas!  Maybe I should.  

Somewhat related is this idea that you can evaluate ideas according to prototypically.  Take the domain content and extract some structure to use as a model.  Mihaly C. says creativity is an interaction between individual, domain and field.  Perhaps you have a combinatorial algorithm that spits out ideas, you have a domain for similar ideas, and you need to decide whether the idea is new or not, and whether it's good or not.  How do you predict whether an idea will be good or not?  You need tons of information about the environment.  Perhaps you need to track the most recent idea in the field?  In addition, you need to figure out what "good" means.  

\item \textbf{The Internet as a Brain} If an idea is neurons firing together, as Steven Johnson thinks, then the internet could be a huge source of new ideas, and a way to evaluate them.  The internet is nice because it is constantly being updated by humans.  It is already a centaur.  Lots of people have their info stored in their own brains and transfer portions of it to the web.  

To get a new idea you need to generate it first, and then you need to evaluate it.  The necessary pieces are:

\begin{enumerate}
\item Generate an idea using the internet: choose some term from a list and give it to Google, then scrape the links.
\item Understand the local structure of the idea, what are the clusters, and which clusters are nearby to the cluster of the idea.  You'd need some way, like hyperlinks, to understand the network structure by figuring out which items connected to which other items.  
\item Then do a bunch of network metrics on the result.  Find combinations of ideas that are likely to be: useful (maybe from a calibrated model), new (from a different model) and possible to execute (maybe you'll need humans for this, but you might be able to calibrate a model based on failed ideas and a theory like prototypically.)
\end{enumerate}

\item \textbf{High Throughput Combinatorial Websites} Software is a great `building material' because it's so flexible.  You could possibly use high-throughput on websites for that reason.

\item \textbf{Hybrid Learning Systems} Combine experiments, human judgement, theory, machine learning, structural models, to create a triangulation system that helps you understand truth.  This is a re-hash of other ideas but it might still be worth recording.  Experiments give causality, human judgment can parse everything, machine learning is fast and can do prediction and leverage large datasets, and structural models can do counterfactuals.

\item \textbf{Reverse Waterfall} The seeds in sesame oil float upwards very slowly when you turn the bottle upside down.  You could do this with a lot of stuff.  It's cool.  

\item \textbf{Startup Host Families}  Startup founders need a place to work and often want to live in urban areas.  Maybe if you had a good screening process, some families
would let you stay at their house.  Basically, the same way families host the missionaries but hosting a startup.  No rent payments, food, etc.  In return, the host family gets equity.

\item \textbf{Gecko-Pad In-Ear Headphones} They stick to your ear like a gecko to a window

\item \textbf{Art Subscriptions} You receive a new, large piece of art every quarter.  The business model saves costs since everyone gets the same art, but chosen from a few options. 

\item \textbf{Pinterest + Reddit} Self-evident, I think.

\item \textbf{Curation For Everything} Collaborative filtering and also profile matching (someone like you) can help with discovery.  Also perhaps humans.  Again, I hybrid might work very well.  

\item \textbf{Lay Scholar} Get a large selection of journals, have employees translate them into layman's terms and build a consumer friendly database.  


\end{enumerate}



%----------------------------------------------------------------------------------------
%	Observations
%----------------------------------------------------------------------------------------

\section{Observations}

This is a list of observations about myself that I may find helpful later on.
 
\begin{enumerate}
\item I really love minimalism in objects and spaces, but not necessarily at home.  I like to work in places like that.  I love the idea of attaching functions or activities to places.  I want to work in a minimalist building with high ceilings and talk with colleagues about product design.
\item I love the idea of tools to think new ways and new thoughts:  those asian kids that do crazy math with an abacus, brett victor's ideas, the creativity program I wrote in R.  Convergent, vs. divergent vs. idea manipulation aids. (abacus, random idea generator, algebraic notation.)  Imagine a 3D graph of your profits, and you could play endlessly with different variables.  
\item \textbf{Hybrid Systems} Two components both working on their best thing and trading off.  Carbon fiber (the carbon fabric has awesome tensile strength but isn't rigid, the resin adds the rigidity),  Gasoline + Electricity, Apple's Fusion Drive.  Are alloys considered hybrid systems?  At the molecular or atomic scale, maybe.  At the human scale, probably not.  
\item \textbf{Density Meets Convexity} If a function is increasing and convex, then the returns to scale are fantastic.  It's just plain old convex programming I guess, and maybe this observation isn't all so insightful, but it just strikes me as crazy how often you see that a major improvement was made on a material or object because someone figured out a new way to arrange some sub-structures more densely.  Micro chips are a great example.  Moore's law is driven by the ability of engineers to cram more transistors onto a chip.  It's such a huge, important consequence of one small thing.
\item \textbf{Integration} The benefits of integration vs. decentralization seem sort of complex.  You have more control with integration but also more obstacles.  Coase's ideas seem quite relevant: as long as the transaction cost in a decentralized world exceed the costs in an integrated one, the firm will exist.  A major transaction cost for both Apple and Tesla was the vision.  Getting a contractor to share the vision was too hard.  Maybe this means that if you face exacting constraints, you favor integration.  Otherwise, you farm out the work.  Or, to use the words of a saying ``if you want something done right, you have to do it yourself.''  This is probably why firms won't go away until we can communicate super well.  
\end{enumerate}

\bibliographystyle{ormsv080}

\bibliography{ideas}


\end{document}

































