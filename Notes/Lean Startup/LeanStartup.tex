\documentclass[11pt, oneside]{article}   	% use "amsart" instead of "article" for AMSLaTeX format
\usepackage{geometry}                		% See geometry.pdf to learn the layout options. There are lots.
\geometry{letterpaper}                   		% ... or a4paper or a5paper or ... 
%\geometry{landscape}                		% Activate for for rotated page geometry
%\usepackage[parfill]{parskip}    		% Activate to begin paragraphs with an empty line rather than an indent
\usepackage{graphicx}				% Use pdf, png, jpg, or eps with pdflatex; use eps in DVI mode
								% TeX will automatically convert eps --> pdf in pdflatex		
\usepackage{amssymb}

\title{The Lean Startup}
\author{Book by Eric Ries, Notes by Jason Bell}
%\date{}							% Activate to display a given date or no date

\begin{document}
\maketitle
\section{Definition of a Startup}

Ries' definition of a startup is \begin{quote} A startup is a human institution designed to create a new product or service under conditions of extreme uncertainty.  \end{quote}

Startups exist to \textit{learn.}  The goal is to learn how to build a sustainable business.  Redefined this way, the startup can leverage the scientific method.

\section{BEL Loop}

The Build-Experiment-Learn feedback loop is at the core of the Lean Startup.

\subsection{Hypotheses}

There are at least two major hypotheses to test for every startup, namely:
\begin{enumerate}
\item \textbf{Value Hypothesis.}  Why customers will want to pay (and \textit{that} they will want to pay. 
\item \textbf{Growth Hypothesis} How the product or service will spread.
\end{enumerate}

 \subsection{MVP}
 
 The minimum viable product is the mechanism for moving through the BEL loop quickly.  Ries gives a few categories of MVPs.  The goal is not to generate quick revenue, it is to generate quick \textit{learning.}
 
 \begin{enumerate}
 \item \textbf{The Video MVP.} Drew Houston made a video to show what Dropbox \textit{could} be like, and people loved it.  It didn't even work yet.
 \item \textbf{The Concierge MVP.} Food on the Table got a single customer and gave her white-glove service.  The CEO went to her house every week to provide their service.  
 \end{enumerate} 
 \subsection{Experiment}
 
 These are two examples.  There are many more.  The absolute requirement is not that the MVP match up to some ideal in the designer's mind.  Rather it must do a good job of testing an assumption about the consumer.  For example, will anyone pay for a vegetable smoothie?


\subsection{Experiment}

Split-testing and cohort format for data are two major points.  They allow you to isolate the effect of your actions.

\section{Pivoting}

Pivoting is anchoring in one element of the product and changing the other stuff.  It is a new hypothesis, so like any other, it should be tested with a new MVP and experiments.  Several pivot categories are:

\begin{enumerate}
\item \textbf{Zoom-In Pivot} `What previously was considered a single feature in a product becomes the whole product.'
\item \textbf{Zoom-Out Pivot} `What was considered the whole product becomes a single feature of a much larger product.'
\item \textbf{Customer Segment Pivot.}  `The company realizes that the product it is building solves a real problem for real customers but that they are not the type of customers it originally planned to serve.  [... T]he product hypothesis is partially confirmed, solving the right problem, but for a different customer than originally anticipated.'  
\item \textbf{Customer Need Pivot.}  As a result of getting to know customers, the company learns that the original problem isn't important, but they find a new problem that is.
\item \textbf{Platform Pivot.}  `A change from an application to a platform or vice-versa.'
\item \textbf{Business Architecture Pivot.}  From niche to mass or the reverse.
\item \textbf{Value Capture Pivot.} `There are many ways to capture the value a company creates.  These methods are referred to commonly as monetization or revenue models.  These terms are much too limiting.  Implicit in the idea of monetization is that it is a separate ``feature'' of a product that can be added or removed at will.  In reality, capturing value is an intrinsic part of the product hypothesis.  Often changes to the way a company captures value can have far-reaching consequences for the rest of the business, product, and marketing strategies.'  
\item \textbf{Engine of Growth Pivot.}  3 very common engines are viral, sticky, and paid growth models.
\item \textbf{Channel Pivot.}  Refers to distribution.
\item \textbf{Technology Pivot.} Same problem, same customers, new method for solving the problem. 
\end{enumerate}

\section{Speed}

\subsection{Batch Size}

A batch refers to how many units of a product move from one production stage to the next.  Small batch sizes are great for faster learning.  They often bring system costs down, even if unit costs increase.  Ries doesn't say this, but there are minimum efficient batch sizes, I'm certain.  I think the point is not to get the smallest batch size possible, but to notice that most batch sizes are way too big, and marginal improvements could be made by shrinking them.

\subsection{Growth}

There are 4 very common mechanisms by which growth occurs, and 3 engines that drive growth using the 4 mechanisms.  First, the 4 mechanisms.

\begin{enumerate}
\item \textbf{Word of mouth.}  
\item \textbf{As a side effect of product usage.}  Fashion is an example.
\item \textbf{Funded advertising.}
\item \textbf{Repeat purchase or use.}
\end{enumerate}  

The 3 engines are

\begin{enumerate}
\item \textbf{Sticky.}  Comes from keeping customers.
\item \textbf{Viral.} Spreading by use.  Different from Word of Mouth because they don't have to decide to do anything, it just happens.  Like fashion.  
\item \textbf{Paid. }  Plowing revenue back into customer acquisition.  If the customer acquisition costs are low enough, relative to the lifetime values, the company grows.  Otherwise growth is unsustainable. 
\end{enumerate}

The important thing about the engines of growth are that they can be expressed as equations, and so you can measure, say, the viral coefficient, and experiment until you figure out what pushes it up.  This takes it from a fluffy management theory to a well-specified methodology.


\end{document}  