%%%%%%%%%%%%%%%%%%%%%%%%%%%%%%%%%%%%%%%%%
% Short Sectioned Assignment
% LaTeX Template
% Version 1.0 (5/5/12)
%
% This template has been downloaded from:
% http://www.LaTeXTemplates.com
%
% Original author:
% Frits Wenneker (http://www.howtotex.com)
%
% License:
% CC BY-NC-SA 3.0 (http://creativecommons.org/licenses/by-nc-sa/3.0/)
%
%%%%%%%%%%%%%%%%%%%%%%%%%%%%%%%%%%%%%%%%%

%----------------------------------------------------------------------------------------
%	PACKAGES AND OTHER DOCUMENT CONFIGURATIONS
%----------------------------------------------------------------------------------------

\documentclass[paper=a4, fontsize=11pt]{scrartcl} % A4 paper and 11pt font size

\usepackage[T1]{fontenc} % Use 8-bit encoding that has 256 glyphs
\usepackage{fourier} % Use the Adobe Utopia font for the document - comment this line to return to the LaTeX default
\usepackage[english]{babel} % English language/hyphenation
\usepackage{amsmath,amsfonts,amsthm} % Math packages
\usepackage{amscd}
\usepackage{amssymb}

\usepackage{color}
\definecolor{light-gray}{gray}{0.95}

\usepackage{lipsum} % Used for inserting dummy 'Lorem ipsum' text into the template

\usepackage{sectsty} % Allows customizing section commands
\allsectionsfont{\centering \normalfont\scshape} % Make all sections centered, the default font and small caps

\usepackage{graphicx} % For graphics

\usepackage{listings} % For displaying code
\usepackage{courier}
%\DeclareGraphicsRule{.tif}{png}{.png}{`convert #1 `dirname #1`/`basename #1 .tif`.png}
\lstset{
         basicstyle=\footnotesize\ttfamily, % font is monospace and small
         %numbers=left,              
         numberstyle=\tiny,          % size of numbers
         %stepnumber=2,               
         numbersep=5pt,              
         tabsize=2,                  
         extendedchars=false,         
         breaklines=true,            
         keywordstyle=\color{red},
         stringstyle=\color{white}\ttfamily, 
         showspaces=false,           % white spaces
         showtabs=false,             % no character when tab is pressed
         xleftmargin=25pt,
         framexleftmargin=17pt,
         framexrightmargin=5pt,
         framexbottommargin=4pt,
         %backgroundcolor=\color{lightgray},
         showstringspaces=false      % don't show spaces    
 }


\usepackage{fancyhdr} % Custom headers and footers
\pagestyle{fancyplain} % Makes all pages in the document conform to the custom headers and footers
\fancyhead{} % No page header - if you want one, create it in the same way as the footers below
\fancyfoot[L]{} % Empty left footer
\fancyfoot[C]{} % Empty center footer
\fancyfoot[R]{\thepage} % Page numbering for right footer
\renewcommand{\headrulewidth}{0pt} % Remove header underlines
\renewcommand{\footrulewidth}{0pt} % Remove footer underlines
\setlength{\headheight}{13.6pt} % Customize the height of the header

\numberwithin{equation}{section} % Number equations within sections (i.e. 1.1, 1.2, 2.1, 2.2 instead of 1, 2, 3, 4)
\numberwithin{figure}{section} % Number figures within sections (i.e. 1.1, 1.2, 2.1, 2.2 instead of 1, 2, 3, 4)
\numberwithin{table}{section} % Number tables within sections (i.e. 1.1, 1.2, 2.1, 2.2 instead of 1, 2, 3, 4)

\setlength\parindent{0pt} % Removes all indentation from paragraphs - comment this line for an assignment with lots of text

%----------------------------------------------------------------------------------------
%	TITLE SECTION
%----------------------------------------------------------------------------------------

\newcommand{\horrule}[1]{\rule{\linewidth}{#1}} % Create horizontal rule command with 1 argument of height
%the below reassigns the function \texttt{} the name "code".  So instead of typing \texttt{blah blah} you can %type \code{blah blah}.  It's just easier.

\newcommand{\code}[1]{\texttt{#1}}

\title{	
\normalfont \normalsize 
%\textsc{The University of Iowa} \\ [25pt] % Your university, school and/or department name(s)
\horrule{0.5pt} \\[0.4cm] % Thin top horizontal rule
\huge Ideas Round 2 \\ % The assignment title
\horrule{2pt} \\[0.5cm] % Thick bottom horizontal rule
}

\author{J. Jason Bell} % Your name

\date{\normalsize\today} % Today's date or a custom date

\begin{document}

\maketitle % Print the title



\section{Introduction}

I tried to categorize these ideas according to aspects that have implications for feasibility.  In each idea I included a \textit{growth hypothesis} and a \textit{value hypothesis.}  These terms come from `The Lean Startup,' by Eric Ries.  The first is a statement of belief about how the product will spread, and the second is a belief that the product is valuable and why.  I believe that in order to succeed, these hypotheses must be confirmed to a comfortable degree of certainty.  

Another point is that many of these ideas are not startups per se, but small businesses instead.  The difference is that a small business is unlikely to scale past a certain level.  I think this is okay, for me, for now.
\section{Ideas That Use Our Existing Skills}

\begin{enumerate}
\item \textbf{Videos for Startups.}  We find startups and sell videos of whatever they want, but I'm imagining product demonstration videos.  Go to getbulb.com, or search for the video associated with a product called `automatic link' to see an example of what I imagine.  We can add value by helping them create quality media.  \textit{Growth Hypothesis:}  Startups will like our videos and tell other startups.  In addition, we can seek them out and sell. \textit{Value Hypothesis:}  Startups will be willing to pay enough for these videos to keep us profitable.  They will pay for the videos because they understand media have a powerful effect on consumer perceptions of product quality, but they don't want to learn tons more skills or acquire costly equipment.   

\item \textbf{Ethnographic Videos.}  An ethnography is a tool used for creating new products.  For example, Swiffer is the result of an ethnography, along with many other well known and successful products.  The basic idea is that you watch consumers while they use a product to try and see things in their behavior that aren't easily verbalized.  This technique is very powerful, and essentially unused by small businesses.  We could try and target small businesses. \textit{Growth Hypothesis:}  Word-of-mouth, and perhaps repeat use (for multi-product small businesses.) \textit{Value Hypothesis:}  This is tough.  Small businesses may not be able to see the power of this technique.  We may need to do a lot of convincing in the beginning.  It is likely that we only need to get a few clients in order to succeed, since the technique has a very high success rate.  

\item \textbf{A Media Recommendation Service.}  There are many recommendation services that rely on algorithms.  These are designed for a broad market.  What I have in mind is a niche market.  We provide extremely high quality, customized recommendations for customers.  I can see this working for books, music, movies, and TV shows.  We could start out by getting a single client and giving them a concierge style service in order to learn more about the growth and value prospects of the idea.  As an example, imagine an employee of Goldman Sachs in SLC, say John.  He is busy with work all of the time, but is making quite a lot of money.  He likes Pandora but the recommendations leave much to be desired, so he hires us.  We find out which bands he likes, find something new for him, and as part of the price we gift the songs to his iTunes account.  Maybe it could be set up as a subscription, or maybe as a three month consultation where we recommend, get feedback, recommend again, feedback again, and so forth until we unlock a large body of suitable media.  Essentially, we would be personal curators for pretty well off people. \textit{Growth Hypothesis:}  I think this would spread by word of mouth.  \textit{Value Hypothesis:}  I find myself often searching for new movies, music, or TV shows that I like and are clean.  I think recommendations from Amazon, Pandora, etc. are okay, but a person could do a lot better.  I think there are more people like me.


\item \textbf{Julia's Purses.}  I think we should revisit this idea.  Julia already has a product that seems to have promise.  If we all pitch in, we might be able to get her to sustainable profitability.  At that point we all could talk about either scaling it up (building a brand and creating a portfolio of products), taking some sort of payout and leaving Julia to run it, or selling it.  If we do this, we get the bonus of a fourth startup member without having to dig.\textit{Growth Hypothesis:}  Word-of-mouth, paid growth (SEO, trade shows, retail channels), and viral growth (people see someone with the purse, they like it, they buy it).  \textit{Value Hypothesis:}  People buy purses already, so all that is left is to see if this purse can compete for attention and budget.  Lucky for us, all we have to do to find out is try selling it.


\item \textbf{Garden Builders.}  This isn't a startup, just a small business. We could learn all about the best ways to grow food at home.  Logan already knows a lot about this.  We wouldn't build new technology, we'd just set gardens up for clients.  They pay us for the hydroponics, greenhouses, etc. that we use, and we set it up in the lowest maintenance way possible.  Along these lines, we could create all manner of businesses setting up specialized things for people for a small fee.  Like professional painters, but with much rarer skills.  \textit{Growth Hypothesis:}  Word-of-mouth, paid growth and viral growth (awesome garden, how'd you get it?)  \textit{Value Hypothesis:}  People want gardens but don't want to get them set up.

\item \textbf{Living Video Game.}  We'd manufacture quests for people.  They'd get a team together, pay an entrance fee, and go through a series of challenges to eliminate other teams.  The prize would be something awesome.  An example: we get a lockbox, fill it with \$100 dollar bills, and hide it somewhere.  Teams who enter the contest all get keys to the lockbox.  The first team to find it gets the money.  On the way, we give out clues to each team, or maybe teams would have to enter challenges to get clues.  We could design it so that teams entering a challenge would have to play against each other in order to get clues (and losing teams are taken out of the running).  It's a lot like a game show, but since you pay to be on, the supply is unrestricted.  We could do all sorts of stuff: paintball fights in the forest, riddles and clues hidden underground or in weird places in the city, interacting with actors hired by us to get information, espionage between teams, etc.  It would be awesome.  The biggest question is time frame (since people are busy), but we could figure that out pretty easily I think.  \textit{Growth Hypothesis:}  Paid growth, word-of-mouth.  Since people would need to get teams together, it would be forced to grow.  We could allow any number of people to form a team, but they'd have to split the winnings evenly, so less incentive to make a huge team.  \textit{Value Hypothesis:}  People like adventure and escaping from everyday stuff.  People like to think of themselves as competent, and they like challenges.  Plus, everyone wants to open a lockbox full of money and get to take it home.  

\end{enumerate}

\section{Vague, Questionable, or Difficult Ideas}

These are the kinds of ideas we said we would avoid this round because they are either ill-formed or things we'll probably have trouble doing.  I still think they are good ideas, so I'll share them just in case.

\begin{enumerate}

\item \textbf{Caring for the elderly.} is an industry that everyone instinctively avoids I think.  I'm pretty sure there are a lot of opportunities there that people just don't want to deal with.  I'm not sure what, however.  Can we get elderly people to do some light work?  It would be good for them and good for society.

\item \textbf{Bite Flosser.}  A device, shaped like a pair of nested U's, and with lots of floss running between the U's.  You bite down on it, so that each space gets a piece of floss, and then you move it forward and backward to simultaneously floss all of your teeth.

\item \textbf{Quirky for Songs.}  People like me have a few half-way formed songs, with melodies, lyrics, etc.  These users would submit their songs in any form they wanted, (most likely performing them on video).  The best songs would be fully created by a professional band.  Song submitters get a small royalty \textit{and} the satisfaction of creating a song, and bands get supplemental tunes; i.e. people do their work for them.
  
\item \textbf{Earthships.}  An earthship is a super-sustainable living structure.  There are a few motivations for building them.  First, they almost always have a green-house or something similar (hydroponics anyone?) with the goal of providing all necessary produce for the occupants.  They are made of very cheap materials such as tires filled with dirt.  They are energy efficient because of the building materials used.  Some earthships cost more than normal homes, but if we learned how to build them we could sell them for cheap (like 10 grand) and then slowly get better and better until we sold them for much more.

\item \textbf{Tailoring Service.}  Most clothes don't fit people very well, but they don't know it.  We'd set up a service, locally first, where customers could send all of their clothes and get custom alterations.  Lots of people search for a long time to find the right fit, but this way, they could buy from almost anywhere.  The nice thing about it is that your measurements would be saved in the service so you could reliably get well fitting stuff.  This type of service already exists, but it mostly deals with newly fabricated items, which are very pricey.  This takes advantage of economies of scale in the textile industry, but achieves results that are nearly as good as custom.  We'd also face a lot of local competition, but I think we could handle it.

\item \textbf{Favor Network.}  This one needs more thought, but it would probably have to be a Facebook app.  People would ask for favors, and help others with favors.  Helping someone with a favor would let you build credits of some kind, and then you could pay for other favors with those credits.  You could also buy credits with money.  An example favors: borrowing a textbook, getting a ride from A to B, using someone's parking spot, an introduction to someone in order to date them, help organizing an apartment before cleaning checks.  The asker would set the price, so the value of doing a favor would be determined by a market, which would save us a ton of work.  All credits would be redeemable for favors, or for money (but if you bought Y bucks worth of credits, you could only sell them back for X dollars, where X < Y, which is how this thing would make money.)  There are networks somewhat like this, but none have broken into the public consciousness yet.  I think the best way to do so would be to focus on one type of task, like introductions for dates or loaning textbooks, much like AirBnB is a specialized network of this kind for allowing people to sleep at your house while you're away.  I believe keeping it tied to Facebook is a good way to catch on quickly.  Once we get into public awareness, the types of favors could be more general.  



\end{enumerate}






\end{document}

































