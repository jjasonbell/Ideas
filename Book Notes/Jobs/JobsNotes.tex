 \documentclass[11pt, oneside]{article}   	% use "amsart" instead of "article" for AMSLaTeX format
 \usepackage{geometry}                		% See geometry.pdf to learn the layout options. There are lots.
 \geometry{letterpaper}                   		% ... or a4paper or a5paper or ... 
 %\geometry{landscape}                		% Activate for for rotated page geometry
 %\usepackage[parfill]{parskip}    		% Activate to begin paragraphs with an empty line rather than an indent
 \usepackage{graphicx}				% Use pdf, png, jpg, or eps with pdflatex; use eps in DVI mode
 								% TeX will automatically convert eps --> pdf in pdflatex		
 \usepackage{amssymb}

 \title{Notes on ``Steve Jobs" by Walter Isaacson}
 \author{J. Jason Bell}
 %\date{}							% Activate to display a given date or no date

 \begin{document}
 \maketitle
 %\section{}
 %\subsection{}

Here I seek to record points worth remembering from the book.  %The first is a list of people by whom Jobs was inspired, and who might also be interesting to read about.

\subsection{People}
\begin{enumerate}
\item Edwin Land (Polaroid)
\item Bill Hewlett and David Packard (HP)
\item Joseph Eichler (Architect) 
\item Alan Kay (Xerox PARC) ``The best way to predict the future is to invent it."  Also, ``People who are serious about software should make their own hardware."
\item Mario Bellini (Architect-Designer)
\item Sergio Pininfarina (Car Maker)
\item Susanna Agnelli (Fiat)
\item I. M. Pei (Japanese Designer)
\item Issey Miyake (Japanese Designer)
\item Raymond Loewy (German Designer)
\item Herbert Bayer (German Designer: Bauhaus)
\item Harmut Esslinger (German Designer: frogleap)
\item Deiter Rams (Braun)
\end{enumerate}

\subsection{Notes}
\begin{enumerate}
\item Mike Markkula's 1-page document ``The Apple Marketing Philosophy" has three points.  I think they are extraordinarily important.  They are the priorities that made Apple as valuable as it is.  Steve Jobs was brilliant at executing these points, almost by nature.  He was smart about the first, and obsessive about the other two.

\begin{enumerate}
\item \textbf{Empathy} ``We will truly understand [the customer's] needs better than any other company."
\item \textbf{Focus} ``In order to do a good job of those things that we decide to do, we must eliminate all of the unimportant opportunities."
\item \textbf{Impute} ``People DO judge a book by its cover.  We may have the best product, the highest quality, the most useful software etc.; if we present them in a slipshod manner, they will e perceived as slipshod; if we present them in a creative, professional manner, we will impute the desired qualities."
\end{enumerate}  

%\item  I really love attaching activities to locations, especially of those locations are nice looking. 

\item There is a principle that has been used at least three times that I can think of to great effect, which is the tight integration of two systems, each used at their most efficient.  Hybrid cars use gasoline at higher speeds where it is more efficient and electric at lower speeds where it is more effective.  The iPod/iMac combo: iMac does the playlist management, which it is best at, and iPod is portable and plays music on the go.  The Fusion drive in iMacs now uses a solid state drive for writing everything, but a large hard disk in case the solid state drive gets full.  The key seems to be automating the transition between the two.  Is this principle descriptive or perhaps generative?  Is it something you just get when working in your industry, or can you create new products using it?

\item Jobs was good at bridging gaps between ``technology and humanities."  As Doug Morris said ``the difference in skill sets between the music folks and technologists is just huge.  This  made Jobs' ability very valuable.  Where else are there large bridges that I could build?

\item I think people have the perception that the iPod was the product that saved Apple.  I think that the iPod would have been worthless without the iTunes store, the iMac, and most importantly, without Jobs' ability to get all the record companies on board.  Every item is crucial, and without Jobs' salesmanship, it would have failed.  Sony's failure is strong evidence of this: they had everything all the way from end-to-end, in fact they were \textit{more} capable of integration than Apple, because they owned record labels too.

His ability to get people on board seemed to be one of the characteristics that set Jobs apart from early on.  Many very talented and smart people got behind Apple and Pixar and helped it work.  

\item Jobs about motivation: \begin{quote} The older I get, the more I see how much motivations matter.  The Zune was crappy because the people at Microsoft don't really love music or art the way we do.  We won because we personally love music.  We made the iPod for ourselves, and when you're doing something for yourself, or your best friend or family, you're not going to cheese out.  If you don't love something, you're not going to go the extra mile, work the extra weekend, challenge the status quo as much.  \end{quote}    

\item Jobs about creativity: \begin{quote} There's a temptation in our networked age to think that ideas can be developed by email and iChat.  That's crazy Creativity comes from spontaneous meetings, from random discussions.  You run into someone, you ask what they're doing, you say `Wow,' and soon you're cooking up all sorts of ideas. \end{quote}    
\end{enumerate}

\end{document}  