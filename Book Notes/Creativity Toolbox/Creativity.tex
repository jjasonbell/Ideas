 \documentclass[11pt, oneside]{article}   	% use "amsart" instead of "article" for AMSLaTeX format
 \usepackage{geometry}                		% See geometry.pdf to learn the layout options. There are lots.
 \geometry{letterpaper}                   		% ... or a4paper or a5paper or ... 
 %\geometry{landscape}                		% Activate for for rotated page geometry
 %\usepackage[parfill]{parskip}    		% Activate to begin paragraphs with an empty line rather than an indent
 \usepackage{graphicx}				% Use pdf, png, jpg, or eps with pdflatex; use eps in DVI mode
 								% TeX will automatically convert eps --> pdf in pdflatex		
 \usepackage{amssymb}

 \title{Creativity}
 \author{J. Jason Bell}
 %\date{}							% Activate to display a given date or no date

 \begin{document}
 \maketitle
 %\section{}
 %\subsection{}

Dyson, Graham, and Jobs all mention a few things about creativity.  They corroborate one another one some points.  Those points follow.

\begin{enumerate}
\item Turn off different kinds of \textbf{filters}.  Graham names the shlep filter and the unsexy filter.  In addition he mentions that people might dismiss ideas as toys.  Dyson urges his readers to be illogical and ignore convention.  Jobs was capable of doing this but didn't offer much commentary about how he did it personally.

\item The importance of \textbf{other people.}  All three either explicitly mention the necessity of depending on other people, or exemplified it in their work.
\item Learn by doing, and invent by doing.  Trying to think of ideas is the worst way to think of ideas.  

\item Jobs and Dyson both stress the importance of \textbf{complete control} over an idea from conception to execution.  This has more to do with execution than creativity.

\item Graham and Dyson mention \textbf{getting to the frontier} of a field in order to make something new.  Dyson says it takes about 6 months, Graham says it takes about a year.  Maybe that is a result of the different contexts in which they were writing.

\item Dyson and Jobs both emphasize \textbf{focus.}  Dyson's motivation seemed to be that users can't process more than one revolutionary advance at a time, and Jobs seemed to care more about the ability to execute something extremely well.  In addition, both meant it in the sense that Deiter Rams meant it: designers should strive to remove unnecessary frills from a product.

\item \textbf{Time.}  Graham says give yourself time, Dyson took time, and Jobs took time.
\item \textbf{Build.}  Graham and Dyson both mention that you have to go try to build things, and that is where the best ideas come from.  Often the first idea is bad but you come up with a better one.
\item \textbf{Figure out what you don't like} about things as they are.  All three mention this.  Dyson came up with the DC1 because he didn't like the Hoover.  Graham suggests that you ``pay particular attention to things that chafe you."  Jobs hated all the junky mp3 players on the market, and phones, and computers.  He hated hardware keyboards, styluses, and ugly things.
\item Focus on the \textbf{needs of the users}, above all else.  Make something that people actually need, and are willing to pay for, even if the user group is small.  Jobs paid a great deal of attention to the users, as did Dyson, as did Graham.  This is all as opposed to thinking about making money and how big the market share could eventually be.
\item Think about \textbf{solving problems}, rather than creating something new.  
\item \textbf{Iterate.}  Everything gets better with iteration.  All three would agree about this.

\end{enumerate}

\end{document}  