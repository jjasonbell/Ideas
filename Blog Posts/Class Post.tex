\documentclass[11pt, oneside]{article}   	% use "amsart" instead of "article" for AMSLaTeX format
\usepackage{geometry}                		% See geometry.pdf to learn the layout options. There are lots.
\geometry{letterpaper}                   		% ... or a4paper or a5paper or ... 
%\geometry{landscape}                		% Activate for for rotated page geometry
%\usepackage[parfill]{parskip}    		% Activate to begin paragraphs with an empty line rather than an indent
\usepackage{graphicx}				% Use pdf, png, jpg, or eps with pdflatex; use eps in DVI mode
								% TeX will automatically convert eps --> pdf in pdflatex		
\usepackage{amssymb}

\title{Class and Income}
\author{J. Jason Bell}
%\date{}							% Activate to display a given date or no date

\begin{document}
\maketitle
%\section{}
%\subsection{}

Class and income are often perceived as the same thing.  Perhaps they are so highly correlated that distinctions between them aren't useful.  I hope to convince you that class and income are not the same, and the difference between them matters.

First, some background on my views of class, which I largely absorbed from an imaginatively titled book, \textit{Class}, by Paul Fussell.  The details in the book are outdated, but the structure of classes has morphed in what I think is a traceable way, so the ideas can be adapted to 2013.  Fussell has 9 categories of class.  I won't name them all, because only 5 are really important nowadays.  I also won't use all of his original names.  Here is the list along with some description:

\begin{enumerate}
\item Upper Class: Lots of inherited money, sometimes recognizable family names, wealthy estates.  You don't know them, and they usually don't want you to.  Somewhere in their family line, someone started a very successful venture, and that person's posterity is living a comfortable life because of it.
\item Upper-Middle Class: They rule the world.  Pres. Obama, every Fortune 500 CEO, everybody with an Ivy League degree, everybody in Silicon Valley, almost every doctor, lawyer, and professor in the world.  They are those whom the blog \textit{Stuff White People Like} and the book \textit{Bobos in Paradise} are about.  If you are reading this blog post, there is a 90\% chance that you are part of the UMC.  The key link between members of this class is achievement: they obsess over it, they worship it, they incorporate it into every activity.  Many members of the UMC are not great thinkers, but they compensate for it by working non-stop.  Being in the UMC isn't synonymous with being liberal, but there is a strong trend that way.\footnote{See \textit{The Big Sort}}
\item Middle Class: Everyone was middle class in the 1950's.  Almost nobody is middle class now.  Middle class people are more conscience of status symbols (`he went to Yale?! Oh my gosh!'), and they are usually more traditional in the sense of the word in 2013.  Members of the middle class are often educated somewhat, but they aren't the top of their class, probably because they didn't grow up thinking they would be.  The discrepancy between income and class is noticeable mostly in this class.  There are very few middle-class people, but there are many middle-income people, even though there are fewer middle income people than there were in the 1950's.  More on this below.  
\item  Blue-Collar:  This term is somewhat meaningless now, but it can be operationally defined.  These people have some education, but not the prestigious kind.  Instead they went to community colleges, or state schools, and never graduate schools.  They start businesses and may have a lot of money, but they aren't particularly educated or cosmopolitan.  Utah has a lot of these people.  Sometimes their parents were members of the next class down, but they pushed their kids to avoid a life of labor.  Sometimes members of the Blue-Collar class have kids who rise to the UMC.  (They seem to skip the middle class altogether, for some reason.)     
\item Working Class:  I hate to borrow the term "working class" because it smacks so of politics---and because politics has mangled its meaning---but it is the best way to bring to mind this group of people.  They are the unskilled laborers of the world.  Members of the UMC focus almost obsessively on helping this class financially.
\end{enumerate}



\end{document}  