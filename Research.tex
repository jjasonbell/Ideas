%%%%%%%%%%%%%%%%%%%%%%%%%%%%%%%%%%%%%%%%%
% Short Sectioned Assignment
% LaTeX Template
% Version 1.0 (5/5/12)
%
% This template has been downloaded from:
% http://www.LaTeXTemplates.com
%
% Original author:
% Frits Wenneker (http://www.howtotex.com)
%
% License:
% CC BY-NC-SA 3.0 (http://creativecommons.org/licenses/by-nc-sa/3.0/)
%
%%%%%%%%%%%%%%%%%%%%%%%%%%%%%%%%%%%%%%%%%

%----------------------------------------------------------------------------------------
%	PACKAGES AND OTHER DOCUMENT CONFIGURATIONS
%----------------------------------------------------------------------------------------

\documentclass[paper=a4, fontsize=11pt]{scrartcl} % A4 paper and 11pt font size

\usepackage[T1]{fontenc} % Use 8-bit encoding that has 256 glyphs
\usepackage{fourier} % Use the Adobe Utopia font for the document - comment this line to return to the LaTeX default
\usepackage[english]{babel} % English language/hyphenation
\usepackage{amsmath,amsfonts,amsthm} % Math packages
\usepackage{amscd}
\usepackage{amssymb}

\usepackage{color}
\definecolor{light-gray}{gray}{0.95}
\usepackage{natbib}

\usepackage{sectsty} % Allows customizing section commands
\allsectionsfont{\centering \normalfont\scshape} % Make all sections centered, the default font and small caps

\usepackage{graphicx} % For graphics

\usepackage{listings} % For displaying code
\usepackage{courier}
%\DeclareGraphicsRule{.tif}{png}{.png}{`convert #1 `dirname #1`/`basename #1 .tif`.png}
\lstset{
         basicstyle=\footnotesize\ttfamily, % font is monospace and small
         %numbers=left,              
         numberstyle=\tiny,          % size of numbers
         %stepnumber=2,               
         numbersep=5pt,              
         tabsize=2,                  
         extendedchars=false,         
         breaklines=true,            
         keywordstyle=\color{red},
         stringstyle=\color{white}\ttfamily, 
         showspaces=false,           % white spaces
         showtabs=false,             % no character when tab is pressed
         xleftmargin=25pt,
         framexleftmargin=17pt,
         framexrightmargin=5pt,
         framexbottommargin=4pt,
         %backgroundcolor=\color{lightgray},
         showstringspaces=false      % don't show spaces    
 }

\usepackage{fancyhdr} % Custom headers and footers
\pagestyle{fancyplain} % Makes all pages in the document conform to the custom headers and footers
\fancyhead{} % No page header - if you want one, create it in the same way as the footers below
\fancyfoot[L]{} % Empty left footer
\fancyfoot[C]{} % Empty center footer
\fancyfoot[R]{\thepage} % Page numbering for right footer
\renewcommand{\headrulewidth}{0pt} % Remove header underlines
\renewcommand{\footrulewidth}{0pt} % Remove footer underlines
\setlength{\headheight}{13.6pt} % Customize the height of the header

\numberwithin{equation}{section} % Number equations within sections (i.e. 1.1, 1.2, 2.1, 2.2 instead of 1, 2, 3, 4)
\numberwithin{figure}{section} % Number figures within sections (i.e. 1.1, 1.2, 2.1, 2.2 instead of 1, 2, 3, 4)
\numberwithin{table}{section} % Number tables within sections (i.e. 1.1, 1.2, 2.1, 2.2 instead of 1, 2, 3, 4)

\setlength\parindent{0pt} % Removes all indentation from paragraphs - comment this line for an assignment with lots of text

\DeclareOldFontCommand{\rm}{\normalfont\rmfamily}{\mathrm}
\DeclareOldFontCommand{\sf}{\normalfont\sffamily}{\mathsf}
\DeclareOldFontCommand{\tt}{\normalfont\ttfamily}{\mathtt}
\DeclareOldFontCommand{\bf}{\normalfont\bfseries}{\mathbf}
\DeclareOldFontCommand{\it}{\normalfont\itshape}{\mathit}
\DeclareOldFontCommand{\sl}{\normalfont\slshape}{\@nomath\sl}
\DeclareOldFontCommand{\sc}{\normalfont\scshape}{\@nomath\sc}
\DeclareRobustCommand*\cal{\@fontswitch\relax\mathcal}
\DeclareRobustCommand*\mit{\@fontswitch\relax\mathnormal}


%----------------------------------------------------------------------------------------
%	TITLE SECTION
%----------------------------------------------------------------------------------------

\newcommand{\horrule}[1]{\rule{\linewidth}{#1}} % Create horizontal rule command with 1 argument of height
%the below reassigns the function \texttt{} the name "code".  So instead of typing \texttt{blah blah} you can %type \code{blah blah}.  It's just easier.

\newcommand{\code}[1]{\texttt{#1}}

\title{	
\normalfont \normalsize 
%\textsc{The University of Iowa} \\ [25pt] % Your university, school and/or department name(s)
\horrule{0.5pt} \\[0.4cm] % Thin top horizontal rule
\huge Idea Repository \\ % The assignment title
\horrule{2pt} \\[0.5cm] % Thick bottom horizontal rule
}

\author{J. Jason Bell} % Your name

\date{\normalsize\today} % Today's date or a custom date

\begin{document}

\maketitle % Print the title

%----------------------------------------------------------------------------------------
%	Introduction 
%----------------------------------------------------------------------------------------

This is a list of ideas.  The first section holds research ideas, or just interesting ideas.  Hopefully, this list will bear fruit in the form of projects that advance my career.\footnote{There are several strategies for generating research ideas.  
\begin{enumerate}
\item Read other research
\item Let it happen by accident
\item Solo pondering
\item Many mechanistic approaches.  For example, maybe I should just think of data sets that are really boring, like John Rust did with bus engines.  Then sell the insights.  
\end{enumerate}
}  The second list is more general, but aimed at business.   


%----------------------------------------------------------------------------------------
%	Research List 
%----------------------------------------------------------------------------------------

\section{Research}

\begin{enumerate}

\item \textbf{Sequential Screening} Like your model, but sequential instead of simultaneous.

\item \textbf{Black Swan Costs} As you scale up, you may be able to sample enough from a distribution of shocks to receive some very large, destabilizing costs, like the VW emissions scandal.  The size of the company makes it hard to tamp down on every single employee.

\item \textbf{Soft Factors in B2B} Vivint beat other companies for naming rights over the Jazz stadium because of soft factors (they will talk about the Jazz more than other companies.)  This is not captured by any economic model I know of.

\item \textbf{Cost Plus Structural Model} A similar idea is below, but this one also seems promising: estimate the average margin set by cost-plus in B2B, then build an optimal model.  The optimal model will generate more profit and can be used for recommendations.  The profit difference could be due to higher costs of implementing the optimal model, and that is an interesting variable "the cost of optimality", maybe.

\item \textbf{Product Color Options} Study a product design choice as simple as how many and which colors to offer.  How big of a difference could it make?  What about the black macbook as a case study?  You give a different color, increase price, and the color takes on an air of exclusivity.  

\item \textbf{Magnitude Studies.}  Catalog tons of important and famous effects and find the magnitude in more realistic contexts.  Yes, all these CB effects exist, but are they big enough to change decisions?  Probably not in many cases. Statistical versus practical significance.   Basically meta-analysis with an edge.

\item \textbf{Imitative Behavior of Firms.}  For dynamic policy functions, maybe I just do what you do, so value functions aren't unique.  That is, they may tend to converge by imitation.  Networks and dynamics come together.  

\item \textbf{The Unexpected Value of Platforms}.  Apple's App Store gave enormous unexpected value to the iPhone.  Large networks add value to complementary products.

\item \textbf{The Value of Graphic Design.}  Check national trade associations of designers for these data, or maybe an enterprising person could collect it.  Find local designers, and clients, and measure revenue, etc. before and afterward.  

\item \textbf{Adjacent Complementarity Across Goods}.  It seems that having more stuff introduces more desire for stuff, because of complementarity.  If you don't have a house, you don't need a couch.  If you only own one car, you only need to fuel one car.  If you don't own an iPod, you probably don't buy music.  The difference between this and other complementarity is that you don't foresee it necessarily at the time of purchase.  Although, the examples above don't support that.  More thought required.

\item \textbf{The Value of a Rival.}  Is it possible that having a very strong rival actually keeps competitors out?  I revise my probability of success downward because I see the extremely fierce competition between Coca-Cola and Pepsi.  Also between Samsung and Apple.  

\item \textbf{Feature competition.}  An entry game in feature space.  Very, very hard, probably. 

\item \textbf{A large scale market simulator.}  You basically build a building, with computers, allow people to learn the rules of the market for a period of time, maybe a week or a month, and then have them engage in market activity, perhaps with real payoffs.  Then you basically reserve time on the market simulator.  You'd use business students as subjects primarily, but anyone who wanted to could use it.

\item \textbf{Visual Brand Spillovers}  Do the brands that are closest to the luxury brands on the shelf seem more or less luxurious?

\item \textbf{Novelty Premiums.}  Newly released golf clubs have huge premiums but don't really have technological improvements to account for the price difference.  What psych/econ model of choice can capture this?  Maybe just a classic case where early buyers have higher valuations, but there seems to be something almost irrational about it, as if the newness causes higher valuations.  
   
\item \textbf{Induced complementarity.}  Generally complementarity is assumed to be a demand primitive, but there are lots of ways to induce complementarity, like bundling.  But product design for this purpose is understudied, I think.  Most papers assume a product design without putting the complementarity parameter under the firm's control.

\item \textbf{Scale Attraction in Unit Sizes.}  Laundry detergent producers like to put higher concentrations in each unit, (2x concentrated).  It takes up less shelf space and you use it faster.  It might be nice to estimate the magnitude of usage rate increase.  Say, if I'm thinking of trying to boost usage rate with a higher concentration of something, will it be worth it?  It depends on costs and competitor reactions.

\item \textbf{Exclusivity in Online Groups.}  Gmail and Birchbox both require users to apply, or be invited by current users in order to use the service.  In a market with network effects, how do companies balance the desire for more customers and the desire to maintain the perception of exclusivity.  In general, exclusivity seems to come from price, but there are some things that are exclusive independently of price, like being the first to see a movie.  This relates to Veblen goods.  It also relates to the quality distribution in the iTunes store vs. Android store.  Android lets everyone in, iTunes filters.  It could be as a function of the prices charged to developers?  Or it could be a screening process.  Screening can improve quality \textit{and} enhance exclusivity perceptions. 

\item \textbf{Product line pruning.}  Ford removed the Ranger from their lineup, in hopes that buyers would move to the F-150.  Is that true?  Myatt and Johnson have a paper about product line pruning that might be applicable. 

\item \textbf{Feature pruning.}  What things can you profitably remove from a car?  This is related to the literature on SKU deletion.  Apparently the removal of redundant items \textit{can} be very good for sales (contrary to what Manzi claims.) (Boatwright and Nunes, 2001).  However, these SKU's were specifically chosen because they are low selling.  The take-away seems to be that it is difficult to remove SKU's but it can possibly increase sales if you do it correctly.  There are a few items to consider in this context. 
	\begin{enumerate}
	\item Overall sales may increase, but what about alienating the few customers who really cared about the feature
	\item Boatwright and Nunes (BN hereafter) consider variety.  The product analog of variety is functionality.  The product analog of ease of decision is usability.   
	\end{enumerate}

\item \textbf{Modeling an ``x-factor."}  This can't just be any error term, it has to sometimes be really big and sometimes be really small.  Tri-modal, maybe?  A mixture of normals?

\item \textbf{Kuhn and Christensen.}  Thomas Kuhn and Clayton Christensen both got major leverage from basically the same idea: progress is mostly incremental, but is interrupted by jumps.  Kuhn's jumps are paradigm shifts, and Christensen's are disruptive technologies.  Perhaps I can apply this idea more deeply in a modeling format.  Basically I can model disruptive innovation.  I'll have to see.  The Lucas Critique is that models don't capture parameters that are deep enough to survive policy changes.  Perhaps even current structural models aren't deep enough to capture changes in innovation.  Could an economic model have predicted the massive adoption rate of tablets?  Probably not.  The models don't provide much understanding of why people have adopted them, even now.  It seems that we must incorporate psychology to truly get it.  

%\item \textbf{Innovation Strategy Comparison}  Two strategies. The first is to face uncertainty about engineering constraints but knowing the market wants a particular product.  The other is to use proven engineering tech, but compromise what you know the market wants a little bit.  Which one wins under which circumstances?  Which one do people use?

%\item \textbf{Catolog of Psych Phenomenon}  Basically a dictionary.  You could compile it from Wikipedia.

\item \textbf{Aggregate Lattitude of Acceptance}  If I could find out when it exists, I could find instant 1\% windfalls.  It must exist in the aggregate though. 

\item \textbf{Product Change Responses to Entry} Lots of price response lit, not much product, even though John Murry would say that's the way to go.

\item \textbf{Pricing Strategy Competition}  Are there circumstances where cost-plus is a best response?  Related to the optimal research.  

\item \textbf{Squatting} your way into the market vs. advertising your way into the market.  I just have to be the only guy around who does X and I can beat a lot of advertising

\item \textbf{Package Sizing Attraction Effects} Read Sanghak's paper about indivisible demand.  The idea here would be to see what sizes firms should make their products to maximize profit, especially fast food firms.  Should you let adults order off the kids' menu?

\item \textbf{Assortment Offerings and Customization}  Where are the industries that might benefit from customization?  Industries that can make the custom part modular (like hemming mass-produced pants), industries with ``alignable'' attributes.  A lot of the problem comes form needing to figure out which size to offer, so make it truly continuous.  Interesting.  You could compare purchases at self-serve places to purchases of the same item when assortments are pre-selected by the producer.  Like all-you-can-eat buffets vs. made-to-order restaurants.  

\item \textbf{Product Vacuums}  You could have different theories about entry into the product markets.  Then you could evaluate each empirically and try to come up with a framework for new product development.  For example, suppose you have a characteristic space and you find pockets that balance emptiness with nearby successful products.  Use other products as a way of `filtering.'  That way maybe you can find arbitrage opportunities: holes in otherwise rich markets have good potential.  If a firm used this entry strategy, maybe it could make some money.

\item \textbf{Endogeneity Impact Tests} Use the BLP model with instruments to ``fix'' the fact that price is correlated with unobserved product attributes: then add a bunch of product attributes to see how it changes things.  You should be able to get to a point where you can say something about the use of the instruments.  

\item \textbf{Pricing Under Demand Uncertainty} Follow \cite{hitsch2006empirical} but instead of investigating new product launch, investigate pricing.  Maybe you can find out that it's optimal to do cost plus, or maybe you'll find, as Hitsch did, that simple decision rules lead to large profit losses.  Bayesian decision theory can act as a guide about how much you should learn about demand.  See \cite{hitsch2006empirical} for references and check to see who cited it, because this has probably been done a lot.

\item \textbf{Price Setting Policies}  Try to model equilibria where firms set prices in different ways: cost plus, competitor-based pricing, and demand-based pricing (the best kind).  Then you can go around and see what pricing evidence suggests is actually happening.  This would be similar to \cite{delossantos} did with search paradigms. 

\item \textbf{Machine Learning and Marketing}  Machine learning methods often use iterative procedures based upon a series of regressions.  If at some point an endogenous variable is entered into the model though, the residual is an inconsistent estimator of the true error and so you can contaminate future estimates.  It is hard to simply guess about this issue though.  Perhaps machine learning methods don't suffer as much from endogeneity.  
	If you can solve a structural model to get  an estimable reduced form, machine learning may have advantages.  However, the reduced form dictates which variables enter the model, and so in the end you may ``throw away'' the value of the machine learning.  
	Counterfactuals are also difficult to think about.  I can predict what would happen, short term, but I'm not sure about longer terms, and unstable situations.  Maybe if you have a training set and a test set with a markedly different environment, you could start to understand how machine learning performs.  Maybe I should compare my VCC model to a machine learning model.
	I should reread ``Uncontrolled.''

\item \textbf{Implications of Price Unawareness} Affluent shoppers don't know the prices of things very much, poor people know very well.  Also, relative prices matter a lot.  The average price of the category should be a big indicator.  Context effects.  Also, expertise can reduce dependence on relative indicators.  

\item \textbf{System 1, System 2 Utility Function} What would it look like?  Could you estimate weights on it?  That could be very cool.

\item \textbf{Deiter Rams Rules of Design} Translate them into math and find out if he's right.  Do it the same spirit as the ``Gut Liking for the Ordinary'' paper.

\item \textbf{Share vs. Profit} Say you have a temporary product advantage.  What will be better: charging a premium until competitors catch up, or acquiring more market share?  Which one will have a better lasting impact?  This could be a big question.

\item \textbf{Dynamic Pricing and Planned Obsolescence}  If a manufacturer can make the product obsolete by design, maybe it changes how they can price.  It dampens competition between generations of products.

\item \textbf{Perceptual Mappings and Physical Attributes}  There are more than one way to get to the same spot in perceptual space.  Not just more than one set of attributes (many-to-one mapping), but also advertising can create perceptions (this brand is perceived as high-quality even though it isn't), and your brand identity can have this influence.  So can price.  What are the most important perceptions to have?  What is the cheapest way to get there?  

\item \textbf{New Forms of Price Discrimination}  You can price discriminate using many things: time (Airline seats, new products), any segmentation variable (age, student or not, fast or slow users), product (size, quality level, format), distribution channel (high-end vs. low-end stores).  It might be interesting from a practitioner's point of view to search the space of variables on which you can price discriminate and then pick the `best' one.  Questions to answer: is price discrimination feasible?  Will it get me a boost in profits?  Will people get mad?   

\item \textbf{Sneaky Price Increases}  Some brands may be able to raise prices without anyone noticing!  Others may be much more committed to prices because of media or their business model: Netflix is doubly committed, because everyone pays a monthly fee and the media will talk about it so much.  Campbell's soup?  Nobody will notice if they just charge 1\% more.

\item \textbf{Tradeoffs Between Certain, Risky, and Uncertain Attributes} Suppose a decision maker sees a car and knows that a) he likes the styling, b) he has a strong idea of the reliability of the car, and c) knows the safety rating but doesn't know his likelihood of getting in an accident.  The first attribute he is certain about, the second he treats as a risky attribute, and the third he doesn't even know the risks.  How will he trade off between these attributes?  He can't even use expected utility on the third one!  Maximin for one attribute, constant for another, expected utility for another?

\item \textbf{Partially Two-Sided Markets} The chicken-and-egg problem arises in two-sided markets: you need side A to entice side B, but you also need side B to entice side A.  Apple solved the two-sided market for smartphone apps accidentally: they got people to like the phone for its own sake.  Then they built a two-sided market on top of their existing user base.  You already have side A all filled out, and the two-sided market works.  Mind Your Major could pull the same strategy.

\item \textbf{Decreasing Marginal Utility of Money}  Under some circumstances it may make sense that money has diminishing marginal utility.  Those circumstances are immediate cash infusions.  My immediate utility of \$100 is probably less than 100 $\times$ my utility of \$1.  However, this is only in the short term.  Eventually you habituate.  I guess the real distinction is short-term vs. long-term.  

\end{enumerate}

\end{document}